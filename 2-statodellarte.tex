\section{Stato dell'arte}
Nel 2012, l'European Journal of Operational Research pubblicò un articolo chiamato \textit{Personnel scheduling: A literature review}\cite{VANDENBERGH2013367}, nel quale un team di sei ricercatori ha raccolto 291 articoli per dare un quadro generale sullo stato dell'arte dal 2004 fino ad allora. Questo capitolo si concentrerà su questa pubblicazione, andando nel dettaglio per alcuni documenti citati e aggiungendo pubblicazioni più recenti. La tassonomia per i problemi di Personnel Scheduling si basa su quattro campi: caratteristiche del personale, vincoli assieme a misure di performance e flessibilità, metodi risolutivi e incorporazione dell'incertezza, area applicativa e applicabilità della ricerca.
\subsection{Caratteristiche del personale}
La prima caratteristica del personale che si definisce è la tipologia di contratto: la maggior parte delle pubblicazioni scientifiche riguardante questo problema trattanto problemi in cui il personale è per intero di tipo full-time. L'articolo di Hojati e Patil \cite{HOJATI201137} è uno dei pochi che va a trattare la categoria dei contratti part-time per intero. Nel caso dei contratti part-time, la gestione dei turni diventa molto più articolata, in quanto bisogna tenere in considerazione nuovi problemi, come la disponibilità dei dipendenti rispetti a giorni specifici o anche la richiesta necessaria per l'azienda. 

In alcuni problemi, le attività possono richiedere alcune skill specifiche, rendendo così il personale considerato ancora più eterogeneo rispetto alla suddivisione "full-time/part-time"; si considerano anche i casi in cui tali attività possono essere svolte da persone che non possiedono le skill necessarie, comportando però un aumento del costo, in quanto saranno ovviamente meno efficienti rispetto al personale qualificato. Le skill specifiche sono in stretto legame con i livelli di produttivitià e anzianità: allo stesso modo della classificazione in base alle skill, un lavoratore meno produttivo equivarrà ad un costo aumentato. Sia la produttività che l'anzianità possono essere combinate assieme, anche con altre skill. In particolare, per l'anzianità possono essere previsti dei privilegi, come giorni di riposo consecutivi o un peso maggiore che viene dato alle loro preferenze per gli orari.

Un'altra classificazione si basa sul raggruppamento degli impiegati: questo è molto utile in problemi dove si considera lo scheduling di un team di persone e non ogni dipendente individualmente, come ad esempio problemi legati all'area dei trasporti, dove si lega il problema di Personnel Scheduling con quello del routing di veicoli. Esempi per questo tipo di problemi sono l'articolo di Sydney C.K. Chu\cite{CHU20071764} o anche l'articolo di Heil, Hoffmann e Buscher\cite{HEIL2020405}, che si preoccupa di inquadrare nel dettaglio i problemi di crew scheduling per il trasporto ferroviario.
%shine on you crazy diamond
Nell'articolo viene anche definita una tassonomia sulla base del tipo di decisioni che devono essere prese (assegnamento delle attività, sequenza dei turni, tempo, altro) e vengono separati in base a se vengono prese per interi team o per singoli membri del personale. Un esempio che evidenzia la tassonomia basata sulle decisioni si ritrova nell'articolo di Horn et al.\cite{Horn01102007}, che si occupa di trovare il modo più efficiente per la Forza di pattugliamento della Royal Australian Navy di utilizzare una nuova generazione di barche militari. Osservando la letteratura, si può notare che la maggior parte degli articoli si occupa solamente di creare una turnazione accettabile o una schedulazione dei lavori per i dipendenti, dato un carico di lavoro deterministico. Quasi mai, questo problema viene integrato con altri problemi di pianificazione, come la pianificazione del personale, previsione e adeguamento della distribuzione del carico di lavoro, distribuzione delle pause, assunzione e licenziamento, formazione,$\cdot$. Diventa importante questo per la ricerca futura: cercare di unire tutte le decisioni in un unico problema di scheduling.

Anche la flessibilità con il tempo ha assunto un ruolo fondamentale, rendendo ancora di più complicata la formulazione dei modelli per i problemi considerati. La flessibilità può riguardare: sovrapponibilità dei turni, diversi orari di inizio, diverse lunghezze dei turni, diverse durate delle pause, $\cdot$. 

Le scelte decisionali prese per lo scheduling diventano fondamentali non solo per questioni economiche, ma anche per la questione psicologica e fisica dei lavoratori. Xu e Hall\cite{XU2021807} pubblicarono un articolo riguardante la fatica lavorativa, introducendola come variabile all'interno dei problemi di scheduling: un lavoratore stanco rende di meno, aumentando di conseguenza il costo complessivo. L'articolo va ad analizzare quanto in letteratura venga misurata e analizzata la fatica e quale sia il suo impatto sulla performance, andando ad analizzare l'impatto che scelte di scheduling possono avere su di essa. Questo tema si lega strettamente al concetto di flessibilità, in quanto uno schema rigido non può rispondere ad una situazione variabile come la fatica di un lavoratore e bisognerà introdurre nuovi concetti, come quello delle rate-modifying activity (RMA, azioni che cambiano la velocità di produzione di una macchina) o delle micro-pause.

\subsection{Vincoli e Obiettivi}

\subsection{Metodi Risolutivi e Incertezza}

\subsection{Aree di Applicazione}

\subsection{AI}