\section{Valori dei coefficienti degli obiettivi}
\begin{center}
	\begin{tabular}{ |c|c|c|c| }
		\hline
		Peso & Primo set & Secondo set & Terzo set\\
		\hline
		$\lambda_1$ & 20.0 & 10.0 & 2.0\\
		$\lambda_2$ & 1.0 & 1.0 & 0.5\\
		$\lambda_3$ & 1.0 & 2.0 & 2.0\\
		$\lambda_4$ & 1.0 & 2.0 & 2.0\\
		$\lambda_5$ & 1.0 & 3.0 & 3.0\\
		$\lambda_6$ & 1.0 & 1.0 & 2.0\\
		$\lambda_7$ & -0.5 & -2.0 & -4.0\\
		$\lambda_8$ & -1.0 & -3.0 & -10.0\\
		$\lambda_{9under}$ & 3.0 & 2.0 & 1.0\\
		$\lambda_{9over}$ & 0.5 & 0.5 & 0.5\\
		\hline
	\end{tabular}
\end{center}
I tre set si differenziano per il peso assegnato alla copertura del fabbisogno. Si va dal primo set, nel quale l'obiettivo quasi esclusivo è quello della copertura, fino al terzo, in cui la copertura rimane comunque il problema principale ma entra in competizione con gli altri obiettivi, come ad esempio quello del riposo continuato. In un contesto reale, l'analisi e lo studio dei coefficienti assegnati agli obiettivi della funzione multiobiettivo rappresentano una fase fondamentale nella risoluzione dei problemi di ottimizzazione multi-obiettivo, in quanto da essa ne deriva la qualità della soluzione ottenuta.