\section*{Introduzione}
\addcontentsline{toc}{section}{Introduzione}
Negli ultimi decenni, i problemi di schedulazione del personale sono stati studiati ampiamente. Una delle cause principali dietro questo interesse sta il lato economico: il costo del lavoro rappresenta ad oggi, per molte realtà aziendali, uno degli elementi principali della spesa aziendale e riuscire a minimizzarlo può portare grossi benefici all'azienda.\\
La prima formulazione di questo problema venne proposta negli anni '50 da George Dantzig e William Edie, ma ad oggi risulta parecchio antiquata, in quanto:
\begin{itemize}
	\item Non esistevano diverse tipologie di contratto, ma solo i contratti full-time.
	\item Ogni dipendente è interscambiabile, non vengono quindi tenute in considerazione le skill del singolo individuo.
	\item Come obiettivo vi era solo quello di minimizzare il costo del personale, non considerando altri fattori come preferenze o indisposizioni. 
\end{itemize}
Ad oggi, i modelli sono diventati molto più complessi e prevedono obiettivi che possono andare in conflitto tra loro, come ad esempio la minimizzazione del costo del personale e le preferenze soggettive del personale. Inoltre, sono aumentati gli strumenti risolutivi a disposizione, variando dai metodi esatti fino alle euristiche: un problema che veniva risolto a mano può venire ora risolto con buoni risultati, non per forza ottimi, velocemente.\\
\subsection{Obiettivo della tesi}
Questa tesi si pone come obiettivo quello di risolvere il problema di personnel scheduling per il reparto di cassa di un supermercato. Il fine ultimo è quello di trovare la miglior pianificazione settimanale possibile per suddetto supermercato e sviluppare uno strumento applicativo software per permettere l'assegnazione dei turni ai dipendenti.
\subsection{Struttura tesi}
Nel Capitolo 1 viene presentato informalmente il problema reale considerato, descrivendone i soggetti coinvolti, le preferenze espresse, ciò che si vuole ottenere e i vincoli di varia natura.

Nel Capitolo 2 viene delineato lo stato dell'arte nel settore, andando a individuare una definizione tecnica precisa per il tipo di problema individuato e andando a descrivere altre eventuali applicazioni degli stessi concetti che andremo a trattare nel corso della tesi.

Nel Capitolo 3 si fornisce la descrizione formale del problema, che verrà tradotta in un vero e proprio modello matematico, riportato all'interno del Capitolo 4.

Nel Capitolo 5 illustra lo strumento utilizzato per l’implementazione del modello matematico (Gurobi), le principali fasi del processo di realizzazione e i risultati dei test sperimentali volti a valutarne l’efficienza e la validità. 

Nel Capitolo 6, una volta terminata la formulazione del piano orario, si definisce il meccanismo di assegnamento dei turni.

Negli appendici infine vengono riportati, per ordine:
\begin{itemize}
	\item Una guida sintetica per l'utilizzo del programma finale, accompagnata da alcuni frammenti di codice commentati.
	\item Valori dei coefficienti di penalità assegnati ai termini della funzione obiettivo.
\end{itemize}



%Il problema così individuato rimanda al problema del \textit{Set Covering}, ove bisogna trovare la più piccola sotto-collezione di set tale per cui l'unione dei suoi elementi sia equivalente all'insieme universo:
%\begin{align*}
%	\min \quad & \sum_{s\in S}x_s \\
%	\text{s.t} & \sum_{s:e\in s}x_s\ge 1, \forall e\in U\\
%	& x_s \in \{0, 1\}
%\end{align*}