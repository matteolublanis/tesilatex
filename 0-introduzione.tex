\section*{Introduzione}
\addcontentsline{toc}{section}{Introduzione}
Negli ultimi decenni, i problemi di schedulazione del personale sono stati studiati ampiamente. Una delle cause principali dietro questo interesse sta il lato economico: il costo del lavoro rappresenta ad oggi, per molte realtà aziendali, uno degli elementi principali della spesa aziendale e riuscire a minimizzarlo può portare grossi benefici all'azienda.\\
La prima formulazione di questo problema venne proposta negli anni '50 da George Dantzig e William Edie, ma ad oggi risulta parecchio antiquata, in quanto:
\begin{itemize}
	\item Non esistevano diverse tipologie di contratto, ma solo full-time.
	\item Ogni dipendente è interscambiabile, non vengono quindi tenute in considerazione le skill del singolo individuo.
	\item L'obiettivo consisteva soltanto nel minimizzare il costo del personale, non considerando altri fattori come preferenze o indisposizioni. 
\end{itemize}
Ad oggi, i modelli sono diventati molto più complessi e prevedono obiettivi che possono andare in conflitto tra loro, ad esempio la minimizzazione del costo del personale e le preferenze del personale. Inoltre, sono aumentati gli strumenti risolutivi a disposizione, variando dai metodi esatti fino ad arrivare alle matheuristics: un problema che veniva risolto a mano può venire ora risolto con risultati migliori in maniera automatica, riuscendo a trovare soluzioni non ottime ma comunque buone in tempi polinomiali.\\
%I problemi di schedulazione si differenziano in tre tipi secondo Baker:
%\begin{itemize}
%	\item Schedulazione dei turni: si programma un piano giornaliero dei turni; il più semplice comprende turni non sovrapponibili (tipico delle compagnie industriali), ma non è abbastanza per gestire situazioni ove il fabbisogno fluttua in intervalli relativamente piccoli (per esempio, call center).
%	\item Schedulazione dei giorni di riposo: al posto di programmare i giorni di lavoro si programmano i giorni di riposo (sia consecutivi che non); problema ricorrente in, per esempio, negozi e ristoranti, ove la settimana lavorativa dura 7 giorni mentre il lavoratore ne lavora solo 5.
%	\item Schedulazione del tour: unione dei due problemi precedenti; la complessità è principalmente influenzata dalla durata minima dell'intervallo di planning (da 15 minuti fino a 8 ore). 
%\end{itemize}
\subsection{Obiettivo}
Questa tesi si pone come obiettivo quello di risolvere il problema di personnel scheduling per il reparto di cassa di un supermercato, il quale (rimarrà anonimo per privacy) ci ha gentilmente fornito una parte dei dati necessari per descrivere l'istanza del problema. Il fine ultimo è quello di trovare la miglior pianificazione settimanale possibile per suddetto supermercato e poi assegnare i turni ai singoli dipendenti.
\subsection{Struttura tesi}
Nel Capitolo 1 METTERE STATO DELL'ARTE.
Nel Capitolo 2 viene presentato informalmente il problema reale considerato, descrivendone i soggetti coinvolti, le preferenze espresse, ciò che si vuole ottenere, i vincoli giuridici e contrattuali.
Nel Capitolo 3 si fornisce la descrizione formale del problema, che verrà tradotta in un vero e proprio modello matematico, specificandone variabili, vincoli e funzioni obiettivo.
Nel Capitolo 4 illustra lo strumento utilizzato per l’implementazione del modello matematico (Gurobi), le principali fasi del processo di realizzazione e i risultati dei test sperimentali volti a valutarne l’efficienza e la validità. 

RIVEDERE ULTIMA PARTE

Viene infine mostrato il procedimento di assegnamento VEDERE SE CAPITOLO 5.
Negli appendici infine vengono riportati, per ordine:
\begin{itemize}
	\item Una guida sintetica per l'utilizzo del programma finale, accompagnata da alcuni frammenti di codice commentati.
	\item Valori dei coefficienti di penalità assegnati ai termini della funzione obiettivo.
\end{itemize}



CAPITOLO 1, 3 4 pagine descrizione verbosa del problema affrontato (riprendere capitolo 2). "In una realtà aziendale ho tot persone..." no notazione matematica. Si vuole definire orario lavorativo\\
Capitolo 2 stato dell'arte\\
Capitolo 3 descrizione del problema: definizione del problema, formulazione del modello matematico\\
Capitolo 4: analisi sperimentale, le istanze che ho generato, da piccole fino a robe più grosse.\\