\section*{Introduzione}
\addcontentsline{toc}{section}{Introduzione}
\subsection{Personnel Scheduling: definizione e caratteristiche}
Un problema di Personnel Scheduling consiste nel trovare il modo migliore per attribuire i turni ai lavoratori, in modo tale sia da rispettare una serie di vincoli di servizio e contrattuali e sia da massimizzare le preferenze del personale riducendo comunque al minimo i costi previsti.

Negli ultimi decenni, i problemi di schedulazione del personale sono stati studiati ampiamente. Una delle cause principali dietro questo interesse sta il lato economico: il costo del lavoro rappresenta ad oggi, per molte realtà aziendali, uno degli elementi principali della spesa aziendale e riuscire a minimizzarlo può portare grossi benefici all'azienda.\\
La prima formulazione di questo problema venne proposta negli anni '50 da George Dantzig e William Edie, ma ad oggi risulta antiquata in quanto:
\begin{itemize}
	\item Non esistevano diverse tipologie di contratto, solo contratti full-time.
	\item Ogni dipendente è interscambiabile, non vengono tenute in considerazione le skill del singolo individuo.
	\item Come obiettivo vi era solo quello di minimizzare il costo del personale, non considerando altri fattori come preferenze o indisposizioni. 
\end{itemize}
Baker fornì una classificazione dei problemi di Personnel Scheduling in tre gruppi:
\begin{itemize}
	\item Shift Scheduling: si pianifica su un orizzonte di pianificazione giornaliero, implicando che il fabbisogno di personale per ciascun turno può essere gestito in modo indipendente per determinare le allocazioni appropriate; rappresenta il problema più facile da risolvere delle tre categorie, ma non risponde a situazioni più complesse (esempio, la necessità dei turni che si sovrappongono, tipico problema per i call center).
	\item Days off Scheduling: al posto di pianificare i turni di lavoro, si pianificano i giorni di riposo per i dipendenti; problema tipico per aziende che lavorano sette giorni su sette, mentre il lavoratore ne lavora solo cinque/sei (una variante del problema potrebbe richiedere che i giorni di riposo siano consecutivi).
	\item Tour scheduling: una combinazione delle due categorie viste precedentemente, bisogna assegnare un tour (giorni della settimana e ore del giorno) da lavorare ad un particolare dipendente. Questa categoria risulta molto più complicata rispetto alle due precedenti e la complessità del problema, seppur dipenda da più fattori, è fortemente influenzata dalla durata minima dell'intervallo di pianificazione (solitamente, dai 15 minuti alle 8 ore).
\end{itemize}
In Ricerca Operativa, il Personnel Scheduling è un problema di tipo NP-hard, o peggio NP-complete: la complessità deriva principalmente da tutte le possibili combinazioni dei turni che si possono ottenere, rendendo la ricerca di un ottimo molto più complessa e lunga.

Il problema è in continua evoluzione e necessita quindi modelli complessi e flessibili a eventuali cambiamenti. Gli strumenti risolutivi oggi disponibili (Gurobi, CPLEX, FICO Xpress, Google OR-Tools) integrano algoritmi sempre più raffinati: un problema che decenni fa veniva approcciato manualmente ottenendo risultati approssimativi, può ora essere risolto con precisione, garantendo soluzioni di alta qualità o, in alcuni casi, ottime a livello globale.\\

\subsection{Obiettivo e struttura della tesi}
Questa tesi si pone come obiettivo quello di risolvere il problema di personnel scheduling (tour scheduling) per il reparto di cassa di un supermercato. Il fine ultimo è quello di trovare la miglior pianificazione settimanale possibile per suddetto supermercato e sviluppare uno strumento applicativo software per permettere un'assegnazione dei turni semplificata ai dipendenti.

Nel Capitolo 1 viene presentato informalmente il problema reale considerato, descrivendone i soggetti coinvolti, le preferenze espresse, ciò che si vuole ottenere e i vincoli di varia natura.

Nel Capitolo 2 viene delineato lo stato dell'arte nel settore, andando a individuare una definizione tecnica precisa per il tipo di problema individuato e andando a descrivere altre eventuali applicazioni degli stessi concetti che andremo a trattare nel corso della tesi.

Nel Capitolo 3 si fornisce la descrizione formale del problema, che verrà tradotta in un vero e proprio modello matematico, riportato all'interno del Capitolo 4.

Nel Capitolo 5 illustra lo strumento utilizzato per l’implementazione del modello matematico (Gurobi), le principali fasi del processo di realizzazione e i risultati dei test sperimentali volti a valutarne l’efficienza e la validità. 

Nel Capitolo 6, una volta terminata la formulazione del piano orario, si definisce il meccanismo di assegnamento dei turni.

Negli appendici infine vengono riportati, per ordine:
\begin{itemize}
	\item Una guida sintetica per l'utilizzo del programma finale, accompagnata da alcuni frammenti di codice commentati.
	\item Valori dei coefficienti di penalità assegnati ai termini della funzione obiettivo.
\end{itemize}