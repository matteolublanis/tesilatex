\section*{Introduzione}
\subsection{Storia, classificazione}
\addcontentsline{toc}{section}{Introduzione}
Negli ultimi decenni, i problemi di schedulazione del personale sono stati studiati ampiamente. Uno delle cause principali dietro questo interesse sta il lato economico: il costo del lavoro rappresenta ad oggi, per molte realtà aziendali, uno degli elementi principali della spesa aziendale e riuscire a minimizzarlo può portare grossi benefici all'azienda.\\
La prima formulazione di questo problema venne proposta negli anni '50 da George Dantzig e William Edie, ma ad oggi risulta parecchio antiquata, in quanto:
\begin{itemize}
	\item Non esistevano diverse tipologie di contratto, ma solo full-time.
	\item Ogni dipendente è interscambiabile, non vengono quindi tenute in considerazione le skill del singolo individuo.
	\item L'obiettivo consisteva soltanto nel minimizzare il costo del personale, non considerando altri fattori come preferenze o indisposizioni. 
\end{itemize}
Ad oggi, i modelli sono diventati molto più complessi e prevedono obiettivi che possono andare in conflitto tra loro, ad esempio la minimizzazione del costo del personale e le preferenze del personale.\\
I problemi di schedulazione si differenziano in tre tipi secondo Baker:
\begin{itemize}
	\item Schedulazione dei turni: si programma un piano giornaliero dei turni; il più semplice comprende turni non sovrapponibili (tipico delle compagnie industriali), ma non è abbastanza per gestire situazioni ove il fabbisogno fluttua in intervalli relativamente piccoli (per esempio, call center).
	\item Schedulazione dei giorni di riposo: al posto di programmare i giorni di lavoro si programmano i giorni di riposo (sia consecutivi che non); problema ricorrente in, per esempio, negozi e ristoranti, ove la settimana lavorativa dura 7 giorni mentre il lavoratore ne lavora solo 5.
	\item Schedulazione del tour: unione dei due problemi precedenti; la complessità è principalmente influenzata dalla durata minima dell'intervallo di planning (da 15 minuti fino a 8 ore). 
\end{itemize}
\subsection{Obiettivo}
Questa tesi si pone come obiettivo quello di risolvere un problema di schedulazione di tour per il roster di cassieri di un supermercato, il quale (rimarrà anonimo per privacy) ci ha gentilmente fornito una parte dei dati necessari per descrivere l'istanza del problema, neanche tutti perché a noi la qualità c'ha rotto il cazzo. Il fine ultimo è quello di trovare la miglior pianificazione settimanale possibile per il supermercato e poi assegnare i turni ai singoli individui.\\
Il processo si suddivide su tre passi:
\begin{enumerate}
	\item Raccolta dei dati.
	\item Modellizzazione del problema in modo corretto e dettagliato. 
	\item Implementazione per trovare la miglior soluzione e verificarne la correttezza.
	\item Assegnare i turni e "automatizzare" il processo (RIVEDERE).
\end{enumerate}
\subsection{Struttura tesi}
Nel capitolo 1 DA DECIDERE SE METTERE STATO DELL'ARTE.\\
Nel Capitolo 1 viene presentato dettagliatamente il problema reale considerato, nel quale vengono descritti informalmente i dati, i vincoli e le funzioni obiettivo.\\
Nel Capitolo 2 si fornisce il necessario per la formulazione del modello matematico, in particolare tramite la programmazione lineare intera. Viene poi fornita una descrizione formale dei dati in input e infine si procede alla formulazione effettiva del modello, spiegando nei dettagli variabili e vincoli confrontandolo con scenari simili trovati in letteratura.\\
Nel Capitolo 3 vengono riportati gli strumenti utilizzati per l'implementazione, le fasi implementative e i risultati dei test che provano l'efficienza e la validità del modello. Viene infine mostrato il procedimento di assegnamento.\\
Negli appendici infine vengono riportati, per ordine:
\begin{itemize}
	\item Alcuni pezzi di codice e guida per l'utilizzo del programma.
	\item Valori dei coefficienti di penalità assegnati ai termini della funzione obiettivo.
\end{itemize}