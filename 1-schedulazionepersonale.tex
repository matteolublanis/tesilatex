\section{Tour scheduling della forza lavoro}
I problemi di tour scheduling della forza lavoro sono un problema frequente per le organizzazioni che lavorano sette giorni su sette, con più di un turno al giorno, come ad esempio hotel, stazioni della polizia, aeroporti o anche negozi. Ai lavoratori è necessario assegnare dei giorni di riposo e anche eventuali ferie o permessi. Bisogna dunque assegnare uno specifico tour (turni e giorni in cui lavorare) nel quale ogni specifico lavoratore deve prestare servizio.\\
La complessità del problema dipende da più fattori, come la presenza di uno o più tipi di contratto, considerazione o meno delle skill del singolo lavoratore, diversi livelli dello stesso tipo di contratto, disponibilità. La durata dell'intervallo di pianificazione minimo però è ciò che più influenza la complessità del problema: utilizzare un intervallo di 8 ore rende il problema molto più semplice rispetto ad un intervallo di soli 15 minuti, il quale aumenta esponenzialmente il numero di variabili da considerare. I problemi si differenziano in continui se il periodo di lavoro giornaliero è di 24 ore, altrimenti è discontinuo se è minore di 24.\\
Vi possono essere altri vincoli lavorativi, come: orario di inizio turno possibili, la minima e massima durata di ogni turno, la frequenza e la durata della pausa pranzo e delle pause, il minimo periodo di riposo tra ogni cambio turno, le ore operative giornaliere, il numero di giorni di lavoro a settimana, limite al numero di giorni lavorativi consecutivi e rotazione dei turni. Altri vincoli potrebbero riguardare scelte più soggettive, come preferenze del personale ed equità nell'assegnamento dei turni.
