\section{Stato dell'arte}
\subsection{Problemi di schedulazione}
I problemi di scheduling riguardano l'allocazione di risorse limitate ad attività nel tempo. Essi richiedono di determinare: l'ordine in cui l'insieme di attività (task) viene eseguito, il tempo in cui ogni attività viene eseguita perseguendo un certo obiettivo e come allocare le risorse che si hanno. Si differenzia lo scheduling dal sequenziamento, il quale si limita al trovare l'ordine migliore con il quale eseguire le attività, senza tenere in conto le tempistiche e l'allocazione delle risorse.\\
I campi di applicazione sono variegati e molteplici:
\begin{itemize}
	\item Shop scheduling: per la schedulazione del lavoro di produzione.
	\item Schedulazione del personale: quello specifico del nostro caso, può riguardare il personale di un qualsiasi ambito; in letteratura è importante il \textit{Nurse Scheduling Problem} (NSP), ovvero il problema di trovare il miglior modo per assegnare le infermiere ai loro turni.
	\item Scheduling di facility: sale operatorie, aule didattiche,\dots
	\item Scheduling di veicoli: spedizioni postali e schedulazione di mezzi pubblici.
	\item Project scheduling.
	\item \dots
\end{itemize}
In generale, i problemi di schedulazione possono essere classificati in base a diversi criteri, tra cui la natura delle risorse, la tipologia delle attività, la presenza di vincoli e l’obiettivo da ottimizzare.
Le risorse possono essere di tipo rinnovabile (ad esempio personale o macchinari disponibili periodicamente) o non rinnovabile (come materiali o budget limitati). Le attività, invece, possono presentare vincoli di precedenza, tempi di esecuzione differenti e requisiti specifici di risorse.

Un altro aspetto fondamentale riguarda la distinzione tra schedulazione statica e schedulazione dinamica:
\begin{itemize}
	\item Statica: tutte le informazioni necessarie sono note in anticipo, e il piano di schedulazione viene elaborato una volta sola.
	\item Dinamica: il piano di schedulazione si può aggiornare in tempo reale, così da adattarsi a eventi imprevisti, come ritardi, guasti o assenze di personale.
\end{itemize}

Gli obiettivi perseguiti possono variare a seconda del contesto applicativo, i più comuni:
\begin{itemize}
	\item minimizzazione del makespan (tempo totale di completamento del lavoro);
	\item minimizzazione dei ritardi o delle penalità associate al mancato rispetto delle scadenze;
	\item massimizzazione dell’utilizzo delle risorse o del throughput produttivo;
	\item equità nella distribuzione del carico di lavoro tra il personale;
	\item rispetto di vincoli normativi o contrattuali (turnazioni, pause, limiti di orario).
\end{itemize}

Dal punto di vista teorico, la maggior parte dei problemi di schedulazione appartiene alla classe dei problemi NP-hard, il che implica che non esistono algoritmi polinomiali noti in grado di risolverli esattamente per istanze di grandi dimensioni. Di conseguenza, nella pratica si ricorre spesso a metodi approssimati, come:
\begin{itemize}
	\item Euristiche costruttive e local search, che generano soluzioni buone in tempi brevi;
	\item Metaeuristiche (ad esempio genetic algorithms, simulated annealing, tabu search, ant colony optimization), capaci di esplorare lo spazio delle soluzioni in modo più efficace;
	\item Metodi di ottimizzazione esatta, come la programmazione lineare intera o la programmazione a vincoli, utilizzati soprattutto per istanze di piccola o media dimensione.
\end{itemize}

Infine, con l’avvento delle moderne tecnologie informatiche, la schedulazione trova applicazione anche in ambiti emergenti come la smart manufacturing, la sanità digitale, i servizi on-demand e il cloud computing, dove l’automazione e l’uso di algoritmi intelligenti (machine learning, sistemi multi-agente, ottimizzazione predittiva) permettono di migliorare l’efficienza e la flessibilità delle decisioni di pianificazione.
\subsection{Schedulazione del personale}
La prima classificazione dei metodi per la schedulazione del personale fu proposta da Baker, il quale divise il personnel scheduling in tre grandi tipologie:
\begin{itemize}
	\item Shift scheduling: anche chiamato time-of-day scheduling, ha come problema principale quello di pianificare i turni giornalieri.
	\item Days off scheduling: la lunghezza della settimana lavorativa non si limita ad andare dal lunedì al venerdì, ma comprende anche sabato e domenica, rendendo così necessario pianificare i giorni di riposo dei lavoratori.
	\item Tour scheduling: combinazione dei due casi precedenti; la complessità è principalmente influenzata dalla durata minima dell'intervallo di planning (da 15 minuti fino a 8 ore). 
\end{itemize}
Un'altra classificazione è quella in base alla soluzione adottata per risolvere il problema. Alfares individuò dieci categorie: soluzione manuale, programmazione intera, modellazione implicita, decomposizione, goal programming, generazione di set di lavoro, soluzioni basate su LP, costruzione/miglioramento, metaeuristiche e altri metodi.

INSERIRE QUALCOSA SU PROBLEMI DI PERSONNEL SCHEDULING, LISTA MAGARI DI PAPERS

