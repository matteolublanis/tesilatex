\section{Tour scheduling della forza lavoro}
I problemi di tour scheduling della forza lavoro sono un problema frequente per le organizzazioni che lavorano sette giorni su sette, con più di un turno al giorno, come ad esempio hotel, stazioni della polizia, aeroporti o anche negozi. Ai lavoratori è necessario assegnare dei giorni di riposo e anche eventuali ferie o permessi. Bisogna dunque assegnare uno specifico tour (turni e giorni in cui lavorare) nel quale ogni specifico lavoratore deve prestare servizio.\\
La complessità del problema dipende da più fattori, come la presenza di uno o più tipi di contratto, considerazione o meno delle skill del singolo lavoratore, diversi livelli dello stesso tipo di contratto, disponibilità. La durata dell'intervallo di pianificazione minimo però è ciò che più influenza la complessità del problema: utilizzare un intervallo di 8 ore rende il problema molto più semplice rispetto ad un intervallo di soli 15 minuti, il quale aumenta esponenzialmente il numero di variabili da considerare. I problemi si differenziano in continui se il periodo di lavoro giornaliero è di 24 ore, altrimenti è discontinuo se è minore di 24.\\
Vi possono essere altri vincoli lavorativi, come: orario di inizio turno possibili, la minima e massima durata di ogni turno, la frequenza e la durata della pausa pranzo e delle pause, il minimo periodo di riposo tra ogni cambio turno, le ore operative giornaliere, il numero di giorni di lavoro a settimana, limite al numero di giorni lavorativi consecutivi e rotazione dei turni. Altri vincoli potrebbero riguardare scelte più soggettive, come preferenze del personale ed equità nell'assegnamento dei turni.\\
Il vincolo comune a tutti questi problemi è quello della copertura del fabbisogno. Il fabbisogno si differenzia in quello minimo e quello ottimale, cambiando il vincolo di copertura da hard a soft. Utilizzare un vincolo di tipo hard non permette una situazione a corto di personale, ma ciò non significa che si debba per forza rispettare il preciso fabbisogno: può esserci una situazione con personale in eccesso, in particolare per poter anche rispondere a situazioni impreviste come malattie o un aumento imprevisto del fabbisogno necessario. L'eccesso o il difetto di personale possono essere più o meno penalizzati, con pesi diversi per ciascuno permettendo così situazioni, per esempio, di difetto di personale rispetto a quelle di eccesso per motivi finanziari.\\
Nel corso degli anni, si è iniziato a tenere sempre di più conto delle preferenze del personale e in particolare si è prestata molta attenzione verso l'equità tra i dipendenti. L'equità si può esprimere in diversi modi in base al contesto nel quale si vuole applicare, per esempio:
\begin{itemize}
	\item Un monte orario simile per tutti i colleghi dello stesso tipo.
	\item In caso di skill diversificate, bisogna affidare un carico del lavoro equo ad ogni dipendente. 
	\item Dare la domenica libera in maniera equa ai dipendenti che lavorano tutta la settimana (alternando magari settimana sì settimana no).
\end{itemize}
Altro argomento di particolare interesse è la struttura dei turni. All'interno del settore terziario la richiesta del personale è molto variegata e può portare a una struttura dei turni insolita e non apprezzata dai dipendenti, ad esempio potrebbero esserci due turni in un singolo giorno con una pausa tra i due di anche di parecchie ore, andando così a coprire sia la mattina che la sera ma vincolando di molto il dipendente al luogo di lavoro, in particolare nel caso in cui abiti lontano.
\subsection{}