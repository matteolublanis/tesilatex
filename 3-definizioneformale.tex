\section{Definizione formale}
In questa sezione, si offre una descrizione formale del problema mediante l'utilizzo di un modello di Programmazione Lineare Mista Intera (MILP), che verrà poi inserito come input in Gurobi, un risolutore matematico. Un modello si struttura nel seguente modo:
\begin{align*}
	\max_{x\in \mathbb{Z}^n} \quad & c^Tx \\
	\text{s.t.} \quad & Ax \leq b, \\
	& x \geq 0
\end{align*}
ove $\max_{x\in \mathbb{Z}^n} c^Tx$ rappresenta la funzione obiettivo che deve essere massimizzata o minimizzata, mentre le altre due righe rappresentano i vincoli del modello. Questa sezione è suddivisa in tre parti:
\begin{enumerate}
	\item Problema base: viene definito formalmente il minimo necessario per avere un modello base funzionante, senza andare incontro a vincoli aggiuntivi come skill o preferenze del personale.
	\item Problema esteso: vengono aggiunti i vincoli mancanti al problema base, analizzando anche quali elementi potrebbero andare in conflitto con quelli precedenti.
	\item Possibili estensioni. 
\end{enumerate}

\subsection{Problema base}

\paragraph{Insieme dei cassieri}
Si definisce l'insieme $C$ come l'insieme dei cassieri. Al momento, i cassieri si differenziano solo per il monte orario previsto da contratto:
\begin{equation*}
	C=FT\cup PT\cup PTW
\end{equation*}
Appartenere a categorie differenti equivale soltanto ad avere monte orari differneti: per il momento, tutti i cassieri condividono la stessa paga oraria (non vi sono differenze di anzianità o di livello contrattuale). Un cassiere $c$ può essere quindi definito nel seguente modo:
\begin{equation*}
	c(Mh, info, altro)
\end{equation*}
ove $Mh$ sta a indicare il monte ore, $info$ rappresenta i dati del cassiere (non utili per la risoluzione ma a livello pratico nell'assegnazione dei turni) e $altro$ che possono essere dati aggiuntivi (skill possedute ad esempio).

\paragraph{Copertura oraria e turni}
Durante una giornata, l'obiettivo del caporeparto è quello di coprire il flusso della clientela in maniera adeguata. Rifacendoci al problema del \textit{Set Covering}, all'articolo di Brucker\cite{BRUCKER2011467} e a quello di Gusmeroli e Bettinelli\cite{gusmeroli2024mixedintegerlinearprogramcreate}, si definiscono:
\begin{center}
	$D=\{1,\dots ,m\}$ i giorni della finestra temporale.\\
	$f_{d,k}$ il fabbisogno per lo slot $k$ del giorno $d$.\\
	$T_d$ l'insieme degli slot temporali del giorno $d$.
\end{center}
Nel caso studiato, $m$ vale 7. Gli slot temporali nei problemi di scheduling possono avere durata variabile, dall'intera giornata lavorativa fino a soli 15 minuti: si è deciso di utilizzare uno slot temporale di mezz'ora. La giornata di lavoro inizia alle 08:00 e finisce alle 20:30 (verrà modificato non appena si introdurranno i vincoli per la spazzatrice), è quindi suddivisa in 25 slot temporali. L'orario è continuato ed è identico per tutti e sette i giorni della settimana, dunque vi sarà da definire un totale di 175 fabbisogni, ognuno per il singolo slot $k$ della giornata $d$. Il fabbisogno viene rappresentato mediante l'utilizzo di una matrice, dove i giorni rappresentano le righe e gli slot le colonne.
I turni assegnati ai dipendenti saranno gli insiemi tipici del \textit{Set Covering}. Si specificano ora i turni:
\begin{center}
	$s(day, start, end, dur)\in S$
\end{center}
$S$ rappresenta l'insieme dei turni $s$, i quali sono caratterizzati da:
\begin{itemize}
	\item day: giorno del turno.
	\item start: slot d'inizio del turno.
	\item end: slot di fine del turno.
	\item dur: durata del turno espressa in quantitativo di slot.
\end{itemize}
Si definisce anche l'insieme $S(d)$, ovvero l'insieme dei turni del giorno $d$.
I turni $s$, come per il fabbisogno, vengono pregenerati. La durata varia dalle 3 alle 5 ore, quindi dai 6 ai 10 slot temporali. Gli slot temporali dovranno essere coperti da almeno $f_{d,k}$ turni. Per definire se un cassiere lavora o meno un certo turno, si usa una variabile decisionale:
\begin{center}
	$x(c,s)\in \{0,1\}$
\end{center}
la quale varrà 1 nel caso in cui al cassiere $c$ sia assegnato il turno $s$, 0 altrimenti.

Si definisce il monte orario:
\begin{center}
	$Mh_c\in \{16,24,48\}$
\end{center}
In quanto sono permessi straordinari, seppur penalizzati, ogni cassiere dovrà lavorare il monte orario previsto fino ad un massimo da stabilire:
\begin{equation}
	Mh_c\le 0.5*\sum_{s\in S}x(c,s)dur(s)\le maxOre_c
\end{equation}
ove $maxOre_c$ può essere stabilito diverso per ogni singolo cassiere oppure essere costante per tutti. Si moltiplica per 0.5 in quanto due slot rappresentano un'ora. Per adesso, viene considerato costante e pari a 48, come descritto nel problema informale: lasciarlo così alto però potrebbe portare problemi indesiderati, come ad esempio l'assegnazione di parecchie ore di straordinari ai dipendenti part-time, per questo più avanti verrà analizzato il problema e verrà penalizzata la situazione di straordinari. 

I cassieri part-time sono gli unici ai quali viene assegnato un singolo turno al giorno, mentre ai full-time e ai part-time fine settimana possono essere assegnati due turni giornalieri:
\begin{displayquote}
	\textit{Part time weekend: 16 ore settimanali, con disponibilità limitata ai giorni di venerdì, sabato e domenica.\dots\\
		\dots ma sono previsti due turni giornalieri per i cassieri full-time e part-time fine settimana (solo il sabato e la domenica per loro)\dots Ai cassieri part-time viene assegnato un solo turno giornaliero.\\
		Un dipendente può lavorare fino a 12 ore al giorno, ma è una situazione indesiderata: si tende ad assegnare fino ad un massimo di 8 ore al giorno ad ogni singolo dipendente\dots}
\end{displayquote}
Si esprimono il relativo vincolo per il numero di turni giornalieri:
\begin{equation}
	0\le \sum_{S(d)}x(c,s)\le j \quad \forall d\in D, \forall c\in C, 
	k =
	\begin{cases}
		2 & \text{se } c\in FT || (c\in PTW \land d\in {sab,dom})\\
		1 & \text{se } c\in PT \\
		0 & \text{altrimenti}
	\end{cases}
\end{equation}
Per il numero di ore giornaliere si esplicita:
\begin{equation}
		0.5*\sum_{S(d)}x_{c,s}dur(s)\le 12 \quad \forall c\in C
\end{equation}
in quanto ogni cassiere può lavorare fino ad un massimo di 12 ore.

Si vuole assicurare che non vengano assegnati due turni che siano in concomitanza l'uno con l'altro per lo stesso cassiere:
\begin{align}
	&x(c,s(d,s_1,e_1, dur_1))\le 1-x(c,s(d,s_2,e_2, dur_2)) \label{nonsovrapposizione}\\
	&\forall c\in C:(e_1\ge s_2||s_1\le e_2)\land (s_1\ge e_2+1||s_2\ge e_1+1)\notag
\end{align}
Per praticità, si definiscono le variabili:
\begin{align*}
	& l(c,d)\in {0,1}\\
	& copre(s,d,k)\in {0,1}
\end{align*}
dove $l(c,d)$ indica se un cassiere $c$ lavora il giorno $d$, mentre $copre(s,d,k)$ indica se il turno $s$ copre lo slot $k$ del giorno $d$. Uno slot $k$ si può considerare coperto dal turno $s$ se il turno inizia prima (o in concomitanza) e finisce dopo (o in concomitanza) dello slot $k$.

Si garantisce la copertura dello slot mediante il seguente vincolo:
\begin{equation}
	\sum_{S(d)}\sum_{c\in C}x(c,s)copre(s,d,k)\ge f(d,k) \quad \forall d\in D, \forall k
\end{equation}

Si può infine introdurre la funzione obiettivo per questa parte:
\begin{equation}
	f_1=\sum_{c \in C}\sum_{s \in S} 0.5 dur(s) x(c,s) 
\end{equation}
ovvero la somma totale delle ore lavorate dall'intero roster dei cassieri, che andrà minimizzata come da richiesta.

\paragraph{Stacco turni}
I cassieri hanno richiesto uno stacco tra due turni giornalieri il più breve possibile. Questo permetterebbe di ridurre al minimo il tempo trascorso sul luogo di lavoro e aumentare la soddisfazione del personale. Il problema si pone solo nel caso in cui il cassiere faccia due turni in un giorno. Si definisce la seguente variabile binaria:
\begin{equation*}
	split(c,s1(d), s2(d)) \in \{0,1\}
\end{equation*}
che indica se al lavoratore $c$ vengono assegnati due turni ($s1$ e $s2$) lo stesso giorno $d$. Deve valere:
\begin{align}
	& split(c,s1(d), s2(d)) \le x(c,s1(d))\\
	& split(c,s1(d), s2(d)) \le x(c,s2(d))\\
	& split(c,s1(d), s2(d)) \ge x(c,s1(d))+x(c,s2(d))-1
\end{align}
per assicurarsi che i turni $s1$ e $s2$ siano stati effettivamente assegnati al cassiere $c$ per il giorno $d$. In quanto ci siamo assicurati della non \hyperref[nonsovrapposizione]{non sovrapposizione}, possiamo definire la variabile di stacco:
\begin{equation*}
	stacco(c,d)\in \mathbb{N}
\end{equation*}
una variabile intera che indica lo stacco tra due turni espresso in numero di slot temporali. Lo stacco sarà determinato da:
\begin{align}
	&stacco(c,d)\ge split(c,s1(d), s2(d))*(start(s2)-end(s1)) \notag \\
	& \forall c\in C, \forall d\in D, \forall s1,s2\in S:day(s1)=day(s2)
\end{align}
che andrà infine minimizzato:
\begin{equation}
	f_2=\sum_{d \in D}\sum_{c \in C} stacco(c,d)
\end{equation}

\paragraph{Giorni di riposo}
Si definisce la variabile:
\begin{equation*}
	riposo(c,d)\in {0,1}
\end{equation*}
che vale 1 se e solo se il cassiere $c$ non è stato assegnato a nessun turno il giorno $d$, 0 altrimenti. Vale quindi:
\begin{equation}
	riposo_{c,d} \ge 1-0.5*\sum_{s\in S_d}x_{c,s}, \quad \forall c\in FT\cup PT, \forall d\in D
\end{equation}
Non si studia il problema per i lavoratori part-time fine settimana, in quanto possono essere assegnati il sabato e la domenica.

Durante la settimana, il cassiere deve avere almeno un giorno di riposo e per evitare un sovraccarico di ore giornaliere (avendo per esempio solo 4 giorni lavorativi da 10 ore ciascuno e 3 di riposo) può arrivare ad un massimo di due giorni di riposo settimanali:
\begin{equation}
	1 \le \sum_{d\in D} riposo_{c,d}\le 2
\end{equation}

\subsection{Problema esteso}

\paragraph{Straordinari}
Gli straordinari vengono penalizzati in quanto prevedono una retribuzione maggiorata, aumentando notevolemente il costo finale per il punto vendita. Si introduce:
\begin{center}
	$O_w\in \mathbb{N}$
\end{center}
che rappresenta le ore straordinarie assegnate al cassiere $w$. Questo valore è maggiore o uguale a 0 e si preferirebbe che valesse 0 per tutti i cassieri del roster. Non vale 0 nel caso in cui:
\begin{align}
	& O_w \ge \sum_{s \in S : day(s)=d} A_{s,k} x_{w,s} - C_{w} 
	&& \forall w \in W \label{vincolo:straordinari}
\end{align}
ovvero quando al cassiere $w$ vengono assegnate più ore di quelle previste dal contratto. La somma di questi valori va minimizzata:
\begin{equation}
	f_3=\sum_{w \in W}O_w
\end{equation}
Non si fanno distinzioni tra supplementari e straordinari: sono entrambi penalizzati allo stesso modo. Il peso assegnato a questo obiettivo può variare in base al periodo, rendendo più favorevoli gli straordinari in periodi più critici, come quelli festivi.

\paragraph{Equità}
I problemi riguardanti l'equità sono:
\begin{itemize}
	\item Carico orario settimanale uguale per tutti i cassieri dello stesso tipo.
	\item Carico orario giornaliero uguale per tutti i cassieri dello stesso tipo, ad esempio scatta una situazione di inequità nel momento in cui un cassiere part-time lavori 6 ore un giorno mentre un altro part-time ne lavora solo 3.
	\item Nel caso dei cassieri full-time e dei part-time, la gestione delle domeniche libere rientra nel principio di equità. 
\end{itemize}
Per il primo punto, si introduce la variabile:
\begin{equation}
	H_w=\sum_{w \in T, \, s} 0.5*L_sx_{w,s}
\end{equation}
che rappresenta le ore totali assegnate al cassiere $w$.

Si definiscono altre due variabili:
\begin{align}
	& d_{w}^{+}\ge 0,\quad d_{w}^{-}\le 0
	&& \forall w\in W\\
	& d_{w}^{+},d_{w}^{-}\in \mathbb{R}
\end{align}
che rappresentano rispettivamente le ore straordinarie oppure quelle mancanti per raggiungere il monte orario previsto da contratto. Vale quindi:
\begin{align}
	&d_{w}^{+}\ge H_w-C_w\\
	&d_{w}^{-}\le H_w-C_w
\end{align}
ove, se le ore assegnate sono di più rispetto a quelle previste, il valore di $d_{w}^{+}$ sarà uguale al surplus orario, mentre varrà il contrario per $d_{w}^{-}$ nel caso in cui venissero assegnate meno ore al cassiere. L'obiettivo è quello di avere uno stesso quantitativo orario per tutti i cassieri dello stesso tipo, quindi si minimizza:
\begin{equation}
	f_4=\sum_{T\in W}\sum_{w\in T} \alpha_1 d_{w}^{+} + \beta_1 d_{w}^{-}
\end{equation}
cioè la somma delle deviazioni dal valore previsto da contratto per ogni tipo. Vale $\alpha \ge 0$, $\beta \le 0$ e in base a quale situazione sia preferibile (di eccesso o di difetto) posso decidere se avere $\alpha \ge |\beta|$ oppure $\alpha < |\beta|$. Nel caso analizzato si considera $\alpha > |\beta|$, in quanto una situazione di surplus è sfavorita per scelta del supermercato; inoltre, per come sono definiti i vincoli attualmente, non è prevista una situazione di difetto di ore rispetto al monte ore.

Altro problema da considerare è la durata dei turni. Sono permessi turni dalle 3 fino a 6 ore, ma una durata eccessiva dei turni non è ben accetta dal personale e potrebbe portare a situazioni di squilibrio, in cui alcuni dipendenti lavorano un solo turno da 6 ore al giorno mentre altri ne lavorano due da 3/4 ore ciascuno. Per questione di praticità, possiamo imporre un'unica preferenza per l'intero personale verso il turno da 4 ore (si potrebbe andare a specificare per ogni dipendente una preferenza specifica). Si introduce:
\begin{align}
	&dur_{s}^{+}\ge 0,\quad dur_{s}^{-}\le 0
	&& \forall s\in S \\
	& dur_{s}^{+}, dur_{s}^{-}\in \mathbb{R}
\end{align}
che indicano la durata di tutti i turni $s$ attivi, aggiungiamo anche:
\begin{align}
	&dur_{s}^{+}\ge 0.5*L_s x_{w,s} - 4.0 \\
	&dur_{s}^{-}\le 0.5*L_s x_{w,s} - 4.0
\end{align}
e, infine, da minimizzare:
\begin{equation}
	f_5=\sum_{s \in S} \alpha_2 dur_{s}^{+} + \beta_2 dur_{s}^{-}
\end{equation}
Ogni cassiere potrebbe esprimere la propria preferenza e si potrebbe risolvere il problema andando a cambiare i parametri appena visti. Come prima, $\alpha \ge 0$, $\beta \le 0$ e $\alpha > |\beta|$, in quanto un turno che superi le 4 ore risulta molto più pesante rispetto ad un turno di 3 ore.
\paragraph{Riposo continuato}
Si introduce una nuova variabile binaria per indicare il riposo continuato per i cassieri full-time:
\begin{gather}
	c_{w,d}\le r_{w,d}\\
	c_{w,d}\le r_{w,d+1}\\
	c_{w,d}\le r_{w,d} + r_{w,d+1} - 1\\
	c_{w,d}\in \{0,1\}, \forall w\in W \setminus PTW, \forall d\in D
\end{gather}
che vale 1 solo nel caso in cui il cassiere $w$ ha due giorni di riposo consecutivi $d$ e $d+1$ durante la settimana. Si vuole massimizzare questa quantità in quanto richiesto dal personale:
\begin{equation}
	f_6=\sum_{w\in FT}\sum_{d\in D}c_{w,d}
\end{equation}

%gimme love
\paragraph{Cassa-reparto}
Fino ad adesso, la situazione di sovracopertura dello slot temporale era ammessa e non veniva considerata. Si introduce:
\begin{align}
	& \Delta_{d,k} = \sum_{w \in W}\sum_{s \in S : day(s)=d} A_{s,k} x_{w,s} - f_{d,k}  
	&&\forall d,k\\
	& \Delta_{d,k}\ge 0
\end{align}
ove $\Delta_{d,k}$ indica di quanto viene superato il fabbisogno. Per come sono definiti i vincoli, $\Delta_{d,k}$ non può essere minore di 0, in quanto una situazione di difetto di personale non è ammessa. Si vuole ora minimizzare la situazione di sovracopertura:
\begin{equation}
	f_7=\sum_{d\in D}\sum_{k\in K_d}\Delta_{d,k}
\end{equation}
Questo è fondamentale, in quanto permette di rendere significativa la differenza tra un cassiere normale e un cassiere che può anche essere assegnato in scatolame. 

Un cassiere che può andare in scatolame avrà un peso minore nel conteggio della copertura del fabbisogno, in quanto, se si presentasse una situazione per cui il numero di cassieri presenti in cassa supera di gran lunga quello necessario, può uscire in corsia, riducendo così il lavoro per il personale di reparto e rimanendo comunque disponibile a rientrare in cassa nel momento in cui il flusso della clientela dovesse aumentare. Si definisce l'insieme di questi lavoratori:
\begin{equation}
	WS\subset W
\end{equation}
I vincoli visti fino ad adesso valgono anche per questo tipo di dipendenti, solo conteranno di meno quando si andrà a considerare il problema della sovracopertura per i motivi appena spiegati:
\begin{align}
	& \Delta_{d,k} = \sum_{w \in W}\sum_{s \in S : day(s)=d} pesoCopertura_w*A_{s,k} x_{w,s} - f_{d,k}
	&& \forall d,k\\
	& pesoCopertura_w=\{1, \rho\}
	&& 0 < \rho < 1
\end{align}

\paragraph{Spazzatrice}
La spazzatrice, per via delle sue dimensioni, può solo essere operata prima dell'apertura dell'ipermercato alla clientela. Il compito viene assegnato principalmente a due cassieri appartenenti al personale interno disponibile, poiché la variabilità del personale addetto alle pulizie, dovuta al fatto che l'ipermercato si affida ad un'azienda esterna, rende più affidabile l’impiego di dipendenti interni. Vengono definiti due nuovi slot per giornata (7:00, 7:30), che possono essere coperti soltanto dal personale formato. Si introduce così una nuova differenza:
\begin{equation}
	WP\subset W,\text{ può appartenere a una qualsiasi categoria contrattuale.}
\end{equation}
Solo i cassieri appartenenti a questo insieme possono essere assegnati agli slot temporali appena definiti:
\begin{equation} 
	\sum_{w \in WP}\sum_{s \in S : day(s)=d} A_{s,k} x_{w,s}\ge f_{d,k} \quad \forall d\in D, k={07:00, 07:30}
\end{equation}
Per via del vincolo del riposo obbligatorio tra due giornate, questi cassieri non possono venire pianificati in chiusura se il giorno dopo iniziano alle 07:00:
\begin{equation}
	A_{s_1,(20:00)} x_{w,s_1}\le 1 - A_{s_2,(07:00)} x_{w,s_2} \quad s_1:day(s_1)=d, s_2:day(s_2)=d+1,\forall d\in (D\setminus {7})
\end{equation}

\subsection{Funzione obiettivo}
\begin{align}
    \min \; & 
    \sum_{i=1}^7\lambda_if_i
    && \lambda_i \in \mathbb{R}
\end{align}
I $\lambda_i$ rappresentano i pesi che noi diamo ai vari obiettivi: più $lambda_i$ è elevato, più l'obiettivo relativo incide nella funzione obiettivo generale, ottenendo di fatto più importanza nella ricerca dell'ottimo. 

\subsection{Possibili estensioni}
\begin{itemize}
	\item I dati possono essere ottenuti mediante l'analisi dei dati delle vendite invece di doversi affidare al lavoro manuale del caporeparto.
	\item La finestra temporale può essere aumentata da una settimana ad un mese per un'eventuale analisi statistica più robusta.
	\item Il modello può essere riutilizzato in altri reparti del supermercato o tutt'altro settore che richieda la schedulazione del personale.
	\item Nel caso in cui il carico di lavoro sia distribuito in maniera equa, si può schedulare il personale in gruppi di persone e non a livello di singolo dipendente; si creano così degli orari che possono circolare in maniera costante (esempio, gruppo 1 assegnato alla mattina mentre il gruppo 2 al pomerigio, mentre la settimana successiva vengono invertiti).
	\item Oltre alle skill, si può introdurre un tasso di produttività, che va ad indicare quanto un dipendente sia produttivo all'interno dell'ipermercato, possibilmente differenziandolo anche in base alla diversa attività che il dipendente riesce a svolgere (un dipendente può essere più produttivo in scatolame rispetto alla cassa, si deve introdurre qualcosa che riesca a definire cosa sia a livello tecnico la produttività, come ad esempio velocità di battitura in cassa).
	\item Il punto vendita potrebbe iniziare ad operare 24/7, rendendo necessario una modifica ai vincoli legati agli orari.
	\item Si possono introdurre differenze di retribuzione dovute a: anzianità, diversi livelli contrattuali, diversi contratti (contratti fatta con vecchia gestione), diverse responsabilità (andare a considerare anche il caporeparto come dipendente da schedulare).
\end{itemize} 