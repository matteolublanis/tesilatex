\section{Definizione formale}
In questa sezione, si offre una descrizione formale del problema mediante l'utilizzo di un modello di Programmazione Lineare Mista Intera (MILP), che verrà poi inserito come input in Gurobi, un risolutore matematico. Un modello si struttura nel seguente modo:
\begin{align*}
	\min/\max_{x\in \mathbb{Z}^n} \quad & c^Tx \\
	\text{s.t.} \quad & Ax \leq b, \\
	& x \geq 0
\end{align*}
ove $\min/\max_{x\in \mathbb{Z}^n} c^Tx$ rappresenta la funzione obiettivo che deve essere massimizzata o minimizzata, mentre le altre due righe rappresentano i vincoli del modello. Questa sezione è suddivisa in due parti:
\begin{enumerate}
	\item Problema base: viene definito formalmente il minimo necessario per avere un modello base funzionante, senza andare incontro a vincoli aggiuntivi come skill o preferenze del personale.
	\item Problema esteso: vengono aggiunti i vincoli mancanti al problema base, analizzando anche quali elementi potrebbero andare in conflitto con quelli precedenti.
\end{enumerate}

\subsection{Problema base}

Si definisce l'insieme $C$ come l'insieme dei cassieri. Al momento, i cassieri si differenziano solo per il monte orario previsto da contratto:
\begin{equation}
	C=FT\cup PT\cup PTW
\end{equation}
Appartenere a categorie differenti equivale soltanto ad avere monte orari differneti: per il momento, tutti i cassieri condividono la stessa paga oraria (non vi sono differenze di anzianità o di livello contrattuale). Un cassiere $c$ può essere quindi definito nel seguente modo:
\begin{equation}
	c(info, variabili)
\end{equation}
ove $info$ rappresenta i dati del cassiere (non utili per la risoluzione, come nome e cognome) e $variabili$ che sono gli effettivi dati utili al problema (monte ore previsto, numero massimo di turni assegnabili giornalieri, skill, etc.).

Durante una giornata, l'obiettivo del caporeparto è quello di coprire il flusso della clientela in maniera adeguata. Rifacendoci al problema del \textit{Set Covering}, all'articolo di Brucker\cite{BRUCKER2011467} e a quello di Gusmeroli e Bettinelli\cite{gusmeroli2024mixedintegerlinearprogramcreate}, si definiscono:
\begin{center}
	$D=\{1,\dots ,m\}$ i giorni della finestra temporale.\\
	$f(d,k)$ il fabbisogno per lo slot $k$ del giorno $d$.\\
	$T(d)$ l'insieme degli slot temporali del giorno $d$.
\end{center}
Nel caso studiato, $m$ vale 7. Gli slot temporali nei problemi di scheduling possono avere durata variabile, dall'intera giornata lavorativa fino a soli 15 minuti: si è deciso di utilizzare uno slot temporale di mezz'ora. La giornata di lavoro inizia alle 08:00 e finisce alle 20:30 (verrà modificato non appena si introdurranno i vincoli per la spazzatrice), è quindi suddivisa in 25 slot temporali. L'orario è continuato ed è identico per tutti e sette i giorni della settimana, dunque vi sarà da definire un totale di 175 fabbisogni, ognuno per il singolo slot $k$ della giornata $d$. Il fabbisogno viene rappresentato mediante l'utilizzo di una matrice, dove i giorni rappresentano le righe e gli slot le colonne.
I turni assegnati ai dipendenti saranno gli insiemi tipici del \textit{Set Covering}. Si specificano ora i turni:
\begin{center}
	$s(day, start, end, dur)\in S$
\end{center}
$S$ rappresenta l'insieme dei turni $s$, i quali sono caratterizzati da:
\begin{itemize}
	\item day: giorno del turno.
	\item start: slot d'inizio del turno.
	\item end: slot di fine del turno. %forse si può togliere
	\item dur: durata del turno espressa in quantitativo di slot.
\end{itemize}
Si definisce anche l'insieme $S(d)\subset S$, ovvero l'insieme dei turni del giorno $d$. Si definisce anche per i turni il valore:
\begin{equation}
	gap(s1, s2)\in \mathbb{Z}_{\ge 0}
\end{equation}
che rappresenta, per ogni coppia di turni $s1$ e $s2$ con $day(s1)=day(s2)$, lo stacco tra i due turni espressa in slot. $gap(s1, s2)$ vale:
\begin{equation}
	gap(s1,s2)=
	\begin{cases}
		0 & \text{se } overlap(s1,s2)=1\\
		start(s2)-end(s1) & \text{se } day(s1)=day(s2)\ \land\ end(s1)\le start(s2)\\
		start(s1)-end(s2) & \text{se } day(s1)=day(s2)\ \land\ end(s2)\le start(s1)
	\end{cases}
\end{equation}
Per indicare se due turni si sovrappongono, si specifica:
\begin{align}
	&overlap(s1,s2)\in \{0,1\}\\
	&overlap(s1,s2)=1 \iff
	&&day(s1)=day(s2)\ \land\ \\
	&
	&&[start(s1)\le end(s2)\ \land\ start(s2)\le end(s1)] \notag
\end{align}
che vale 1 se i turni $s1$ e $s2$ sono sovrapposti. La durata dei turni varia dalle 3 alle 5 ore, quindi dai 6 ai 10 slot temporali. Gli slot temporali dovranno essere coperti da almeno $f_{d,k}$ turni. Per dire che un turno $s$ copre uno slot $k$ del giorno $d$, si usa:
\begin{equation}
	copre(s,d,k)\in {0,1}
\end{equation}
Per definire se un cassiere lavora o meno un certo turno, si usa una variabile decisionale:
\begin{center}
	$x(c,s)\in \{0,1\}$
\end{center}
la quale varrà 1 nel caso in cui al cassiere $c$ sia assegnato il turno $s$, 0 altrimenti.

Si definisce la variabile:
\begin{align}
	&H_{settimana}(c)=0.5*\sum_{s\in S}dur(s)*x(c,s)\\
	&H_{settimana}(c)\in \mathbb{R}
\end{align}
che rappresenta le ore totali assegnate al cassiere $c$. Si moltiplica per 0.5 in quanto due slot rappresentano un'ora.

Ogni cassiere dovrà lavorare il monte orario previsto fino ad un massimale da stabilire:
\begin{align}
	&H_{settimana}(c)\ge Mh(c)
	&&\forall c\\
	&H_{settimana}(c)\le maxOre(c)
	&&\forall c
\end{align}
$maxOre(c)$ può essere definito differentemente per ogni singolo cassiere oppure essere costante per tutti. Per adesso, viene considerato costante e pari a 48, come descritto nella sezione 1. 

I cassieri part-time sono gli unici ai quali viene assegnato un singolo turno al giorno, mentre ai full-time e ai part-time fine settimana possono essere assegnati due turni giornalieri:
\begin{displayquote}
	\textit{Part time weekend: 16 ore settimanali, con disponibilità limitata ai giorni sabato e domenica.\dots\\
	\dots ma sono previsti due turni giornalieri per i cassieri full-time e part-time fine settimana (solo il sabato e la domenica per loro)\dots Ai cassieri part-time viene assegnato un solo turno giornaliero.\\
	Un dipendente può lavorare fino a 12 ore al giorno, ma è una situazione indesiderata: si tende ad assegnare fino ad un massimo di 8 ore al giorno ad ogni singolo dipendente\dots}
\end{displayquote}
Si esprime il relativo vincolo per il numero di turni giornalieri:
\begin{align}
	&0\le \sum_{S(d)}x(c,s)\le maxTurni(c,d)
	&&\forall d, \forall c\\
	&maxTurni(c,d) =
	\begin{cases}
		2 & \text{se } c\in FT || (c\in PTW \land d\in {sab,dom})\\
		1 & \text{se } c\in PT \\
		0 & \text{altrimenti}
	\end{cases}\notag
\end{align}
Per il numero di ore giornaliere assegnate a un cassiere $c$ si definisce:
\begin{align}
	&H_{giorno}(c,d)=0.5*\sum_{S(d)}x(c,s)dur(s)\\
	&H_{giorno}(c,d)\in \mathbb{R}
\end{align}
Deve valere:
\begin{equation}
		H_{giorno}(c,d)\le 12 \quad \forall c, \forall d
\end{equation}
in quanto ogni cassiere può lavorare fino ad un massimo di 12 ore.

Per assegnare due turni ad un cassiere nello stesso giorno, bisogna assicurarsi che i due turni non si sovrappongano:
\begin{align}
	&x(c,s1)+x(c,s2)\le 1
	&&\forall c\in C,\ \forall d\in D,\ \forall s_1<s_2\in S(d): overlap(s_1,s_2)=1	
\end{align}

Si garantisce la copertura di uno slot mediante il seguente vincolo:
\begin{align}
	&\sum_{S(d)}\sum_{c\in C}x(c,s)copre(s,d,k)\ge f(d,k) 
	&&\forall d\in D, \forall k\in T(d)\label{coperturafabbisogno}
\end{align}

Si può infine introdurre la funzione obiettivo per questa parte:
\begin{equation}
	f_1=\sum_{c \in C}H_{settimana}(c)
\end{equation}
ovvero la somma totale delle ore lavorate dall'intero roster dei cassieri, che andrà minimizzata come da richiesta.

I cassieri hanno richiesto uno stacco tra due turni giornalieri il più breve possibile. Questo permetterebbe di ridurre al minimo il tempo trascorso sul luogo di lavoro e aumentare la soddisfazione del personale. Il problema si pone solo nel caso in cui il cassiere faccia due turni in un giorno. Si definisce la seguente variabile binaria:
\begin{equation}
	split(c,d) \in \{0,1\} \quad \forall c\in FT\cup PTW
\end{equation}
che indica se al lavoratore $c$ vengono assegnati due turni lo stesso giorno $d$. Deve valere:
\begin{align}
	&\sum_{s\in S(d)}x(c,s)\le 1+split(c,d)
	&&\forall c\in FT\cup PTW, \forall d\\
	&\sum_{s\in S(d)}x(c,s)\ge 2*split(c,d)
	&&\forall c\in FT\cup PTW, \forall d
\end{align}
In quanto ci siamo assicurati della non sovrapposizione, possiamo definire:
\begin{equation}
	stacco(c,d)\in \mathbb{N}
\end{equation}
una variabile intera che indica lo stacco tra due turni nel giorno $d$ per il cassiere $c$ espresso in numero di slot temporali. Lo stacco sarà determinato da:
\begin{align}
	&stacco(c,d)\le maxGap\cdot split(c,d)\\
	&stacco(c,d)\ge gap(s1,s2)*(x(c,s1)+x(c,s2)-1)
	&&\forall c\in FT\cup PTW, \forall d, \forall (s1,s2)\in S(d),s1\neq s2
\end{align}
che andrà minimizzato:
\begin{equation}
	f_2=\sum_{d \in D}\sum_{c \in FT\cup PTW} stacco(c,d)
\end{equation}

Si definisce la variabile:
\begin{equation*}
	riposo(c,d)\in \{0,1\}
\end{equation*}
che vale 1 se e solo se il cassiere $c$ non è stato assegnato a nessun turno il giorno $d$, 0 altrimenti. Utilizzando un m-vincolo, si definisce la variabile prima spiegata:
\begin{align}
	&\sum_{s\in S(d)}x(c,s)\le maxTurni(c,d)(1-riposo(c,d))
	&&\forall c\in FT\cup PT,\forall d\\
	&\sum_{s\in S(d)}x(c,s)\ge 1-maxTurni(c,d)*riposo(c,d)
	&&\forall c\in FT\cup PT,\forall d
\end{align}
Non si studia il problema per i lavoratori part-time fine settimana, in quanto possono essere assegnati il sabato e la domenica.

Durante la settimana, il cassiere deve avere almeno un giorno di riposo e per evitare un sovraccarico di ore giornaliere (avendo per esempio solo 4 giorni lavorativi da 10 ore ciascuno e 3 di riposo) può arrivare ad un massimo di due giorni di riposo settimanali:
\begin{equation}
	1 \le \sum_{d\in D} riposo(c,d)\le 2 \quad \forall c\in FT\cup PT
\end{equation}

\subsection{Problema esteso}

\paragraph{Straordinari}
Gli straordinari vengono penalizzati in quanto prevedono una retribuzione maggiorata, aumentando notevolemente il costo finale per il punto vendita. Si introduce:
\begin{center}
	$O(c)\in \mathbb{R}_{\ge 0}$
\end{center}
che rappresenta le ore straordinarie assegnate al cassiere $c$. Questo valore è maggiore o uguale a 0 e si preferirebbe che valesse 0 per tutti i cassieri del roster. Non vale 0 nel caso in cui:
\begin{equation}
	O(c) \ge H_{settimana}(c)- Mh(c) \quad \forall c\label{vincolo:straordinari}
\end{equation}
ovvero quando al cassiere $c$ vengono assegnate più ore di quelle previste dal contratto. La somma di questi valori va minimizzata:
\begin{equation}
	f_3=\sum_{c \in C}O(c)
\end{equation}
Non si fanno distinzioni tra supplementari e straordinari: sono entrambi penalizzati allo stesso modo. Il peso assegnato a questo obiettivo può variare in base al periodo, rendendo più favorevoli gli straordinari in periodi più critici, come quelli festivi.

\paragraph{Equità}
I problemi riguardanti l'equità sono:
\begin{itemize}
	\item Carico orario settimanale uguale per tutti i cassieri dello stesso tipo.
	\item Preferenza globale sulla durata dei turni.
	\item Nel caso dei cassieri full-time e dei part-time, la gestione delle domeniche libere rientra nel principio di equità. 
\end{itemize}
Per il primo punto, si definiscono:
\begin{align}
	&\bar{H}_{settimana}(t)\in \mathbb{R}
	&&\forall t\in T(={FT, PT, PTW})\\
	&dev(c)\ge 0
	&&\forall c\in C
\end{align}
dove $\bar{H}_{settimana}(t)$ rappresenta la media delle ore assegnate ai cassieri del tipo $t$, mentre $dev(c)$ sta a indicare la deviazione assoluta per singolo cassiere $c$ dalle ore medie assegnate. Varranno i seguenti vincoli:
\begin{align}
	&\bar{H}_{settimana}(t)=\frac{1}{|t|}\sum_{c\in t}H_{settimana}(c)
	&&\forall t\in T\\
	&dev(c)\ge H_{settimana}(c)-\bar{H}_{settimana}(t)\\
	&dev(c)\ge \bar{H}_{settimana}(t)-H_{settimana}(c)
\end{align}
Si dovrà minimizzare la deviazione totale:
\begin{align}
	&f_4=\sum_{c\in C}dev(c)
\end{align}
Altro problema da considerare è la durata dei turni. Sono permessi turni dalle 3 fino alle 5 ore, ma una durata eccessiva dei turni non è ben accetta dal personale e potrebbe portare a situazioni di squilibrio, in cui alcuni dipendenti lavorano un solo turno da 5 ore al giorno mentre altri ne lavorano due da 3/4 ore ciascuno. Per questione di praticità, possiamo imporre un'unica preferenza per l'intero personale verso il turno da 4 ore (si potrebbe andare a specificare per ogni dipendente una preferenza specifica). Si introduce:
\begin{equation}
	pen(s) =
	\begin{cases}
		p_3 &\text{se } dur(s)=6\\
		p_{3.5} &\text{se } dur(s)=7\\
		0 &\text{se } dur(s)=8\\
		p_{4.5} &\text{se } dur(s)=9\\
		p_5 &\text{se } dur(s)=10
	\end{cases}
\end{equation}
che indica le penalità relative ai tipi di turni separati per durata. Si minimizza quindi:
\begin{equation}
	f_5=\sum_{s \in S, c\in C}pen(s)x(c,s) 
\end{equation}
Questa equità non è intesa come confronto diretto tra cassieri, ma come preferenza globale del personale verso determinate durate di turno; l’introduzione di uno standard comune favorisce una distribuzione più omogenea delle tipologie di turno, senza imporre vincoli di equità individuale. Ogni cassiere potrebbe esprimere la propria preferenza singola, rendendo la penalità dipendente da $c$, ovvero $pen(c,s)$. 

I cassieri full-time e part-time hanno espresso la loro preferenza verso il riposo domenicale. La finestra temporale attuale è:
\begin{equation*}
	D={1,\dots,m}\quad \text{con } m=7
\end{equation*}
La domenica è identificata da $d=7$. Possiamo utilizzare di nuovo la variabile $riposo(c,d)$ e massimizzare così il riposo domenicale:
\begin{equation}
	f_6=\sum_{c\in (FT\cup PT)}riposo(c,7)
\end{equation}

\paragraph{Riposo continuato}
Si introduce una nuova variabile binaria per indicare il riposo continuato per i cassieri full-time:
\begin{equation*}
	riposoContinuato(c,d)\in {0,1}
\end{equation*}
che indica se per due giorni consecutivi il lavoratore è stato a riposo. Per fare valere questa definizione, si specifica:
\begin{align}
	&riposoContinuato(c,d)\le riposo(c,d)
	&&\forall c\in FT, \forall d\in {1,\dots,m-1}\\
	&riposoContinuato(c,d)\le riposo(c,d+1)
	&&\forall c\in FT, \forall d\in {1,\dots,m-1}\\
	&riposoContinuato(c,d)\ge riposo(c,d)+riposo(c,d+1)-1
	&&\forall c\in FT, \forall d\in {1,\dots,m-1}\\
	&\sum_{d=1}^{m-1}riposoContinuato(c,d)\le 1
	&&\forall c\in FT
\end{align}
Si vuole massimizzare questa quantità in quanto richiesto dal personale:
\begin{equation}
	f_7=\sum_{c\in (FT)}\sum_{d\in {1,\dots,m-1}}riposoContinuato(c,d)
\end{equation}

\paragraph{Cassa-reparto}
Fino ad adesso, la situazione di sovracopertura dello slot temporale era ammessa e non veniva considerata. Si introduce:
\begin{align}
	&\Delta(d,k)\in \mathbb{R}_{\ge 0}
\end{align}
che indica di quanto viene superato il fabbisogno $f(d,k)$. $\Delta(d,k)$ non può essere minore di 0, in quanto una situazione di difetto di personale non è ammessa (cambiando i vincoli, si potrebbe introdurre anche questa situazione). Si vuole ora minimizzare la situazione di sovracopertura:
\begin{equation}
	f_8=\sum_{d\in D}\sum_{k\in T(d)}\Delta(d,k)
\end{equation}

Un cassiere che può andare in scatolame avrà un peso minore nel conteggio della sovracopertura del fabbisogno, in quanto, se si presentasse una situazione per cui il numero di cassieri presenti in cassa supera quello necessario, può uscire in corsia, riducendo così il lavoro per il personale di reparto e rimanendo comunque disponibile a rientrare in cassa nel momento in cui il flusso della clientela dovesse aumentare. Si definisce la skill:
\begin{equation}
	c(Mh, info, cassareparto)
\end{equation}
I vincoli visti fino ad adesso valgono anche per questo tipo di dipendenti, solo conteranno di meno quando si andrà a considerare il problema della sovracopertura per i motivi appena spiegati:
\begin{align}
	& \Delta(d,k) \ge \sum_{c \in C}\sum_{s \in S(d)} pesoCopertura(c)*copre(s,d,k)x(c,s) - f(d,k)
	&& \forall d\in D,k\in T(d)\\
	& pesoCopertura(c)=
	\begin{cases}
		\rho_1 & \text{se $c$ non possiede la skill cassa-reparto}\\
		\rho_2 & \text{se $c$ possiede la skill cassa-reparto}
	\end{cases}
	&& \rho_1 = 1, \rho_2 < 1
\end{align}

\paragraph{Spazzatrice}
La spazzatrice, per via delle sue dimensioni, può solo essere operata prima dell'apertura dell'ipermercato alla clientela. Il compito viene assegnato principalmente a due cassieri appartenenti al personale interno disponibile, poiché la variabilità del personale addetto alle pulizie, dovuta al fatto che l'ipermercato si affida ad un'azienda esterna, rende più affidabile l’impiego di dipendenti interni. Vengono definiti due nuovi slot per giornata (prima 1=08:00 e 25=20:00, ora 1=07:00 e 27=20:00), che possono essere coperti soltanto dal personale formato. Si introduce così una nuova categoria:
\begin{align}
	&c(Mh, info, cassareparto, spazzatrice)\\
	&C^{SP}=\{c\in C: c(spazzatrice)=1\}
\end{align}
Si introduce una nuova skill, la quale, se non è presente, non permette la copertura di alcuni slot temporali, in quanto prevede attività diverse da quelle viste fino ad adesso:
\begin{align}
	&x(c,s)=0 
	&&\forall c\notin C^{SP}, \forall d\in D,\\
	&
	&&\forall s\in S(d):(copre(s,d,1)=1\lor copre(s,d,2)=1)\notag
\end{align}
La copertura del fabbisogno per gli slot 1,2 è garantita dallo stesso vincolo di copertura \ref{coperturafabbisogno}. Per via del vincolo del riposo obbligatorio tra due giornate, questi cassieri non possono venire pianificati in chiusura (ultimo slot) se il giorno dopo iniziano alle 07:00. Si pregenera un insieme delle coppie di turni che non possono essere pianificati assieme per un singolo cassiere:
\begin{align}
	&I^{11}=\{(s1,s2):day(s2)=day(s1)+1\land \text{riposo tra $s1$ e $s2$ inferiore alle 11 ore}\}
\end{align}
Varrà di conseguenza:
\begin{align}
	&x(c,s_1)+x(c,s_2)\le 1 
	&&\forall (s_1,s_2)\in I^{11},\forall c\in C\\
\end{align}
In quanto la finestra temporale è limitata ad una singola settimana, il problema del controllo del rispetto del vincolo delle 11 ore di riposo tra la domenica e il lunedì può essere spostato dalla definizione del modello al problema dell'assegnamento dei turni, oppure, per evitare di introdurre vincoli su orizzonti esterni alla finestra di pianificazione, può essere risolto impedendo l'assegnazione del cassiere addetto alla pulizia con la spazzatrice allo slot della chiusura della domenica:
\begin{equation}
	x(c,s)=0 \quad \forall s\in S: copre(s,7,27)=1, \forall c\in C^{SP}
\end{equation}
Il vincolo appena trovato è una semplificazione, che potrebbe inoltre introdurre delle inequità dal punto di vista delle assegnazioni dei turni, in quanto i cassieri che possono passare la spazzatrice non avranno mai il turno di chiusura la domenica.

\subsection{Funzione obiettivo}
\begin{align}
    \min \; & 
    \sum_{i=1}^8\lambda_if_i
    && \lambda_i \in \mathbb{R}
\end{align}
I $\lambda_i$ rappresentano i pesi che noi diamo ai vari obiettivi: più $\lambda_i$ è elevato, più l'obiettivo relativo incide nella funzione obiettivo generale, ottenendo di fatto più importanza nella ricerca dell'ottimo. 
