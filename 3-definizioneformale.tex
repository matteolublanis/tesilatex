\section{Definizione formale}
In questa sezione, si offre una descrizione formale del problema mediante l'utilizzo di un modello di Programmazione Lineare Mista Intera (MILP), che verrà poi inserito come input in Gurobi, un risolutore matematico. Un modello si struttura nel seguente modo:
\begin{align*}
	\max_{x\in \mathbb{Z}^n} \quad & c^Tx \\
	\text{s.t.} \quad & Ax \leq b, \\
	& x \geq 0
\end{align*}
ove $\max_{x\in \mathbb{Z}^n} c^Tx$ rappresenta la funzione obiettivo che deve essere massimizzata o minimizzata, mentre le altre due righe rappresentano i vincoli del modello. Questa sezione è suddivisa in tre parti:
\begin{enumerate}
	\item Problema base: viene definito formalmente il minimo necessario per avere un modello base funzionante, senza andare incontro a vincoli aggiuntivi come skill o preferenze del personale.
	\item Problema esteso: vengono aggiunti i vincoli mancanti al problema base, analizzando anche quali elementi potrebbero andare in conflitto con quelli precedenti.
	\item Possibili estensioni. 
\end{enumerate}
\subsection{Problema base}

\paragraph{Insieme dei cassieri}
Si indica con $w$ un cassiere generico appartenente al roster. I cassieri si differenziano tra loro principalmente per gli orari contrattuali assegnati, introduciamo gli insiemi:
\begin{center}
	$W=FT\cup PT \cup PTW$\\
	$FT=\{w_1,...w_n\}$, full-time\\
	$PT=\{w_1,...w_p\}$, part-time\\
	$PTW=\{w_1,...w_k\}$, part-time weekend\\
	$FT\cap PT = \O$\\
	$FT\cap PTW = \O$\\
	$PT\cap PTW = \O$
\end{center}
ove $W$ rappresenta l'insieme di tutti i cassieri. Per adesso, le uniche differenze che vi sono tra i diversi cassieri sostano solo nel monte orario: appartenere ad un insieme, equivale a condividere la stessa paga oraria (stesso inquadramento lavorativo e stessa tipologia di contratto, nessuna anzianità) e lo stesso minimo di ore da lavorare con tutti gli altri dipendenti appartenenti allo stesso sottoinsieme.

\paragraph{Copertura oraria e turni}
Durante una giornata, l'obiettivo del caporeparto è quello di coprire il flusso della clientela in maniera adeguata. Rifacendoci al problema del \textit{Set Covering} e all'articolo di Brucker\cite{BRUCKER2011467}, si definiscono:
\begin{center}
	$D=\{lun, \dots , dom\}=\{1,\dots ,7\}$ l'insieme dei giorni settimanali, che rappresenta la finestra temporale (da 0, lunedì alle 7, fino all'ultimo slot, domenica alle 20).\\
	$f_{d,k}$ il fabbisogno per lo slot $k$ del giorno $d$.\\
	$K_d$ l'insieme degli slot del giorno $d$.
\end{center}
Gli slot temporali possono avere durata variabile, dall'intera giornata lavorativa fino a soli 15 minuti: si è deciso di utilizzare uno slot temporale di mezz'ora. La giornata di lavoro inizia alle 08:00 e finisce alle 20:30 (verrà modificato non appena si introdurranno i vincoli per la spazzatrice), è quindi suddivisa in 25 slot temporali. L'orario è continuato ed è identico per tutti e sette i giorni della settimana, dunque vi sarà da definire un totale di 175 fabbisogni, ognuno per il singolo slot $k$ della giornata $d$.
I turni assegnati ai dipendenti saranno gli insiemi tipici del \textit{Set Covering}. Specifichiamo:
\begin{center}
	$S=\{s_1, \dots , s_l\}$ l'insieme dei turni $s$
\end{center}
I turni $s$ sono caratterizzati da:
\begin{itemize}\label{defturno}
	\item $day(s)$: il giorno $d$ del turno $s$.
	\item $start(s)$: lo slot di inizio del turno $s$.
	\item $L_s$: durata ($\in \{6,7,\dots, 12\}$, misurata in slot) del turno $s$.
	\item $A_{s,k}$: copertura dello slot $k$ dal turno $s$ (1 se coperto, 0 altrimenti).
\end{itemize}
Gli slot temporali dovranno essere coperti da almeno $f_{d,k}$ turni. Per definire se un cassiere lavora o meno un certo turno, usiamo una variabile decisionale:
\begin{center}
	$x_{w,s}\in \{0,1\}$
\end{center}
la quale varrà 1 nel caso in cui al cassiere $w$ sia assegnato il turno $s$, 0 altrimenti. Unendola alla matrice di copertura $A_{s,k}$ possiamo così dire se un cassiere $w$, prendendosi in carico il turno $s$, copra o meno lo slot $k$.

I cassieri si differenziano per il monte orario minimo in base alla tipologia di contratto:
\begin{center}
	$C_{w}\in \{20,24,48\}$
\end{center}
Una volta introdotto il necessario per l'orario e la copertura, si può iniziare a definire i vincoli relativi: 
\begin{align}
	& \sum_{w \in W}\sum_{s \in S : day(s)=d} A_{s,k} x_{w,s} 
	\ge f_{d,k} 
	&& \forall d, k \label{vincolo:copertura}\\
	& C_{w} \le \sum_{w \in W, s} 0.5*L_sx_{w,s} \le 48
	&& \text{monte ore} \label{vincolo:monteorept}
\end{align}
La copertura del fabbisogno si specifica mediante il primo vincolo. Questo vincolo assicura che almeno $f_{d,k}$ cassieri siano presenti nello slot temporale $k$ del giorno $d$. Con questo vincolo non è ammissibile una situazione di difetto di personale, ma è ammissibile un eccesso di personale. Si potrebbe ammettere una situazione di difetto modificando il valore di $f_{d,k}$ o cambiando il peso del vincolo (introducendo ad esempio una nuova variabile $p_f$, il cui valore può ridurre o aumentare $f_{d,k}$).

Ogni cassiere deve lavorare almeno il monte orario minimo definito da contratto, ma può raggiungere le 48 ore facendo straordinari.

Gli ultimi vincoli riguardano il numero di turni giornalieri, che si ricorda sono:
\begin{displayquote}
	\textit{Part time weekend: 20 ore settimanali, con disponibilità limitata ai giorni di venerdì, sabato e domenica.\dots\\
	\dots ma sono previsti due turni giornalieri per i cassieri full-time e part-time fine settimana (solo il sabato e la domenica per loro)\dots Ai cassieri part-time viene assegnato un solo turno giornaliero.\\
	Un dipendente può lavorare fino a 12 ore al giorno, ma è una situazione indesiderata: si tende ad assegnare fino ad un massimo di 8 ore al giorno ad ogni singolo dipendente\dots}
\end{displayquote}
Si definiscono i seguenti vincoli:
\begin{align}
	& \sum_{w \in FT, \, s : day(s)=d} x_{w,s} \le 2, 
	&& \forall d \label{vincolo:ft}\\
	& \sum_{w \in PT, \, s : day(s)=d} x_{w,s} \le 1, 
	&& \forall d \label{vincolo:pt}\\
	& x_{w\in PTW,s} = 1, 
	&& day(s)=5 \land start(s)=16{:}30 \label{vincolo:weekend}\\[6pt]
	& x_{w\in PTW,s} = 0, 
	&& day(s)<5 \label{vincolo:weekend2}\\
	& \sum_{w \in PTW, \, s : day(s)=d} x_{w,s} \le 2, 
	&& d>5\label{vincolo:weekend3}
\end{align}
I primi due vincoli rappresentano i vincoli per i cassieri full-time e part-time, mentre gli ultimi tre sono i vincoli per i part-time fine settimana. Per richiesta del caporeparto, i cassieri part-time fine settimana il venerdì lavorano solo il turno della chiusura (dalle 16:30 alle 20:30), mentre il sabato e la domenica vengono trattati come cassieri full-time.

Si può infine introdurre la funzione obiettivo per questa parte:
\begin{equation}
	f_1=\sum_{w \in W}\sum_{s \in S} 0.5 L_s x_{w,s} 
\end{equation}
ovvero la somma totale delle ore lavorate dall'intero roster dei cassieri, che andrà minimizzata come da richiesta.

\paragraph{Stacco turni}
I cassieri hanno richiesto uno stacco tra due turni giornalieri il più breve possibile. Questo permetterebbe di ridurre al minimo il tempo trascorso sul luogo di lavoro e aumentare la soddisfazione del personale. Il problema si pone solo nel caso in cui il cassiere faccia due turni in un giorno:
\begin{align}
	& split_{w,d} \ge \sum_{s : day(s)=d} x_{w,s} - 1\\
	& split_{w,d} \in \{0,1\}
\end{align}
è una variabile decisionale, che ci dice se il cassiere $w$ è assegnato a due turni $s_1$ e $s_2$ nello stesso giorno $d$.

I turni $s$, per come sono \hyperref[defturno]{definiti}, permettono il calcolo dello slot di fine turno:
\begin{equation}
	end(s)=start(s)+L(s)
\end{equation}
e da questo si può trovare il valore dello stacco:
\begin{equation}
	stacco_{w,d} \ge  split_{w,d} \cdot (start(s_2) - end(s_1)), \quad \forall w \in W
\end{equation}
che viene considerato solo nel caso in cui effettivamente il cassiere abbia due turni lo stesso giorno. Devono valere di conseguenza anche:
\begin{align*}
	& x_{w,s_1} = x_{w,s_2} = 1,\\
	& day(s_1) = day(s_2),\\
	& start(s_2) > end(s_1). 
\end{align*}
Bisogna infine minimizzare:
\begin{equation}
	f_2=\sum_{d \in D}\sum_{w \in W} stacco_{w,d}
\end{equation}
ovvero gli stacchi di tutti i cassieri.

\paragraph{Giorni di riposo}
Si considera giorno di riposo un qualsiasi giorno della settimana nel quale il cassiere non è stato assegnato a nessun turno:
\begin{equation}
	r_{w,d} \ge 1-\sum_{s\in S_d}x_{w,s}, \quad \forall w\in W \setminus PTW,\forall d\in D
\end{equation}
Durante la settimana, il cassiere deve avere almeno un giorno di riposo e per evitare un sovraccarico di ore giornaliere può arrivare ad un massimo di due giorni di riposo settimanali:
\begin{equation}
	1 \le \sum_{d\in D} r_{w,d}\le 2
\end{equation}
Per i part-time fine settimana il problema non si pone, in quanto durante la settimana non gli si può assegnare turni.

\subsection{Problema esteso}
\paragraph{Straordinari}
Gli straordinari vengono penalizzati in quanto prevedono una retribuzione maggiorata, aumentando notevolemente il costo finale per il punto vendita. Si introduce:
\begin{center}
	$O_w\in \mathbb{N}$
\end{center}
che rappresenta le ore straordinarie assegnate al cassiere $w$. Questo valore è maggiore o uguale a 0 e si preferirebbe che valesse 0 per tutti i cassieri del roster. Non vale 0 nel caso in cui:
\begin{align}
	& O_w \ge \sum_{s \in S : day(s)=d} A_{s,k} x_{w,s} - C_{w} 
	&& \forall w \in W \label{vincolo:straordinari}
\end{align}
ovvero quando al cassiere $w$ vengono assegnate più ore di quelle previste dal contratto. La somma di questi valori va minimizzata:
\begin{equation}
	f_3=\sum_{w \in W}O_w
\end{equation}
Non si fanno distinzioni tra supplementari e straordinari: sono entrambi penalizzati allo stesso modo. Il peso assegnato a questo obiettivo può variare in base al periodo, rendendo più favorevoli gli straordinari in periodi più critici, come quelli festivi.

\paragraph{Equità}
I problemi riguardanti l'equità sono:
\begin{itemize}
	\item Carico orario settimanale uguale per tutti i cassieri dello stesso tipo.
	\item Carico orario giornaliero uguale per tutti i cassieri dello stesso tipo, ad esempio scatta una situazione di inequità nel momento in cui un cassiere part-time lavori 6 ore un giorno mentre un altro part-time ne lavora solo 3.
	\item Nel caso dei cassieri full-time e dei part-time, la gestione delle domeniche libere rientra nel principio di equità. 
\end{itemize}
Per il primo punto, si introduce la variabile:
\begin{equation}
	H_w=\sum_{w \in T, \, s} 0.5*L_sx_{w,s}
\end{equation}
che rappresenta le ore totali assegnate al cassiere $w$.

Si definiscono altre due variabili:
\begin{align}
	& d_{w}^{+}\ge 0,\quad d_{w}^{-}\le 0
	&& \forall w\in W\\
	& d_{w}^{+},d_{w}^{-}\in \mathbb{R}
\end{align}
che rappresentano rispettivamente le ore straordinarie oppure quelle mancanti per raggiungere il monte orario previsto da contratto. Vale quindi:
\begin{align}
	&d_{w}^{+}\ge H_w-C_w\\
	&d_{w}^{-}\le H_w-C_w
\end{align}
ove, se le ore assegnate sono di più rispetto a quelle previste, il valore di $d_{w}^{+}$ sarà uguale al surplus orario, mentre varrà il contrario per $d_{w}^{-}$ nel caso in cui venissero assegnate meno ore al dipendente. L'obiettivo è quello di avere uno stesso quantitativo orario per tutti i cassieri dello stesso tipo, quindi si minimizza:
\begin{equation}
	f_4=\sum_{T\in W}\sum_{w\in T} \alpha_1 d_{w}^{+} + \beta_1 d_{w}^{-}
\end{equation}
cioè la somma delle deviazioni dal valore previsto da contratto per ogni tipo. Vale $\alpha \ge 0$, $\beta \le 0$ e in base a quale situazione sia preferibile (di eccesso o di difetto) posso decidere se avere $\alpha \ge |\beta|$ oppure $\alpha < |\beta|$. Nel caso analizzato si considera $\alpha > |\beta|$, in quanto una situazione di surplus è sfavorita per scelta del supermercato; inoltre, per come sono definiti i vincoli attualmente, non è prevista una situazione di difetto di ore rispetto al monte ore.

Altro problema da considerare è la durata dei turni. Sono permessi turni dalle 3 fino a 6 ore, ma una durata eccessiva dei turni non è ben accetta dal personale e potrebbe portare a situazioni di squilibrio, in cui alcuni dipendenti lavorano un solo turno da 6 ore al giorno mentre altri ne lavorano due da 3/4 ore ciascuno. Per questione di praticità, possiamo imporre un'unica preferenza per l'intero personale verso il turno da 4 ore (si potrebbe andare a specificare per ogni dipendente una preferenza specifica). Si introduce:
\begin{align}
	&dur_{s}^{+}\ge 0,\quad dur_{s}^{-}\le 0
	&& \forall s\in S \\
	& dur_{s}^{+}, dur_{s}^{-}\in \mathbb{R}
\end{align}
che indicano la durata di tutti i turni $s$ attivi, aggiungiamo anche:
\begin{align}
	&dur_{s}^{+}\ge 0.5*L_s x_{w,s} - 4.0 \\
	&dur_{s}^{-}\le 0.5*L_s x_{w,s} - 4.0
\end{align}
e, infine, da minimizzare:
\begin{equation}
	f_5=\sum_{s \in S} \alpha_2 dur_{s}^{+} + \beta_2 dur_{s}^{-}
\end{equation}
Ogni cassiere potrebbe esprimere la propria preferenza e si potrebbe risolvere il problema andando a cambiare i parametri appena visti. Come prima, $\alpha \ge 0$, $\beta \le 0$ e $\alpha > |\beta|$, in quanto un turno che superi le 4 ore risulta molto più pesante rispetto ad un turno di 3 ore.
\paragraph{Riposo continuato}
Si introduce una nuova variabile binaria per indicare il riposo continuato per i cassieri full-time:
\begin{gather}
	c_{w,d}\le r_{w,d}\\
	c_{w,d}\le r_{w,d+1}\\
	c_{w,d}\le r_{w,d} + r_{w,d+1} - 1\\
	c_{w,d}\in \{0,1\}, \forall w\in W \setminus PTW, \forall d\in D
\end{gather}
che vale 1 solo nel caso in cui il cassiere $w$ ha due giorni di riposo consecutivi $d$ e $d+1$ durante la settimana. Si vuole massimizzare questa quantità in quanto richiesto dal personale:
\begin{equation}
	f_6=\sum_{w\in FT}\sum_{d\in D}c_{w,d}
\end{equation}

%gimme love
\paragraph{Cassa-reparto}
Fino ad adesso, la situazione di sovracopertura dello slot temporale era ammessa e non veniva considerata. Si introduce:
\begin{align}
	& \Delta_{d,k} = \sum_{w \in W}\sum_{s \in S : day(s)=d} A_{s,k} x_{w,s} - f_{d,k}  
	&&\forall d,k\\
	& \Delta_{d,k}\ge 0
\end{align}
ove $\Delta_{d,k}$ indica di quanto viene superato il fabbisogno. Per come sono definiti i vincoli, $\Delta_{d,k}$ non può essere minore di 0, in quanto una situazione di difetto di personale non è ammessa. Si vuole ora minimizzare la situazione di sovracopertura:
\begin{equation}
	f_7=\sum_{d\in D}\sum_{k\in K_d}\Delta_{d,k}
\end{equation}
Questo è fondamentale, in quanto permette di rendere significativa la differenza tra un cassiere normale e un cassiere che può anche essere assegnato in scatolame. 

Un cassiere che può andare in scatolame avrà un peso minore nel conteggio della copertura del fabbisogno, in quanto, se si presentasse una situazione per cui il numero di cassieri presenti in cassa supera di gran lunga quello necessario, può uscire in corsia, riducendo così il lavoro per il personale di reparto e rimanendo comunque disponibile a rientrare in cassa nel momento in cui il flusso della clientela dovesse aumentare. Si definisce l'insieme di questi lavoratori:
\begin{equation}
	WS\subset W
\end{equation}
I vincoli visti fino ad adesso valgono anche per questo tipo di dipendenti, solo conteranno di meno quando si andrà a considerare il problema della sovracopertura per i motivi appena spiegati:
\begin{align}
	& \Delta_{d,k} = \sum_{w \in W}\sum_{s \in S : day(s)=d} pesoCopertura_w*A_{s,k} x_{w,s} - f_{d,k}
	&& \forall d,k\\
	& pesoCopertura_w=\{1, \rho\}
	&& 0 < \rho < 1
\end{align}

\paragraph{Spazzatrice} \textbf{RIVEDERE}
La spazzatrice, per via delle sue dimensioni, può solo essere operata prima dell'apertura dell'ipermercato alla clientela. Il compito viene assegnato principalmente a due cassieri appartenenti al personale interno disponibile, poiché la variabilità del personale addetto alle pulizie, dovuta al fatto che l'ipermercato si affida ad un'azienda esterna, rende più affidabile l’impiego di dipendenti interni. Vengono definiti due nuovi slot per giornata (7:00, 7:30), che possono essere coperti soltanto dal personale formato. Si introduce così una nuova differenza:
\begin{equation}
	WP\subset W
\end{equation}
Solo i cassieri appartenenti a questo insieme possono essere assegnati agli slot temporali appena definiti:
\begin{equation} 
	\sum_{w \in WP}\sum_{s \in S : day(s)=d} A_{s,k} x_{w,s}\ge f_{d,k} \quad \forall d\in D, k={07:00, 07:30}
\end{equation}
Per via del vincolo del riposo obbligatorio tra due giornate, questi cassieri non possono venire pianificati in chiusura se il giorno dopo iniziano alle 07:00:
\begin{equation}
	A_{s_1,(20:00)} x_{w,s_1}\le 1 - A_{s_2,(07:00)} x_{w,s_2} \quad s_1:day(s_1)=d, s_2:day(s_2)=d+1,\forall d\in (D\setminus {7})
\end{equation}

\subsection{Funzione obiettivo}
\begin{align}
    \min \; & 
    \sum_{i=1}^7\lambda_if_i
    && \lambda_i \in \mathbb{R}
\end{align}
I $\lambda_i$ rappresentano i pesi che noi diamo ai vari obiettivi: più $lambda_i$ è elevato, più l'obiettivo relativo incide nella funzione obiettivo generale, ottenendo di fatto più importanza nella ricerca dell'ottimo. 

\subsection{Possibili estensioni}
\begin{itemize}
	\item I dati possono essere ottenuti mediante l'analisi dei dati delle vendite invece di dovere affidarsi al lavoro manuale del caporeparto.
	\item La finestra temporale può essere aumentata da una settimana ad un mese per un'eventuale analisi statistica più robusta.
	\item Il modello può essere riutilizzato in altri reparti del supermercato o tutt'altro settore che richieda la schedulazione del personale.
	\item Nel caso in cui il carico di lavoro sia distribuito in maniera equa, si può schedulare il personale in gruppi di persone e non a livello di singolo dipendente; si creano così degli orari che possono circolare in maniera costante (esempio, gruppo 1 assegnato alla mattina mentre il gruppo 2 al pomerigio, mentre la settimana successiva vengono invertiti).
	\item Oltre alle skill, si può introdurre un tasso di produttività, che va ad indicare quanto un dipendente sia produttivo all'interno dell'ipermercato, possibilmente differenziandolo anche in base alla diversa attività che il dipendente riesce a svolgere (un dipendente può essere più produttivo in scatolame rispetto alla cassa, si deve introdurre qualcosa che riesca a definire cosa sia a livello tecnico la produttività, come ad esempio velocità di battitura in cassa).
	\item Il punto vendita potrebbe iniziare ad operare 24/7, rendendo necessario una modifica ai vincoli legati agli orari.
	\item Si possono introdurre differenze di retribuzione dovute a: anzianità, diversi livelli contrattuali, diversi contratti (contratti fatta con vecchia gestione), diverse responsabilità (andare a considerare anche il caporeparto come dipendente da schedulare).
\end{itemize} 