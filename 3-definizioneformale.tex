\section{Definizione formale}
\subsection{Problema base}
\paragraph{Insieme dei lavoratori}
Indichiamo con $w$ un dipendente generico appartenente al roster. I dipendenti si differenziano tra loro principalmente per gli orari contrattuali assegnali, introduciamo quindi gli insiemi:
\begin{center}
	$W=FT\cup PT \cup PTW$\\
	$FT=\{w_1,...w_n\}$, full-time\\
	$PT=\{w_1,...w_p\}$, part-time\\
	$PTW=\{w_1,...w_k\}$, part-time weekend\\
	$FT\cap PT = \O$\\
	$FT\cap PTW = \O$\\
	$PT\cap PTW = \O$\\
\end{center}
ove $W$ rappresenta l'insieme di tutti i dipendenti. Per adesso, le uniche differenze che vi sono tra i diversi $w$ sostano solo nel monte orario: appartenere ad un insieme, equivale a condividere la stessa paga oraria (stesso inquadramento lavorativo e stessa tipologia di contratto, nessuna anzianità) e lo stesso minimo di ore da lavorare con tutti gli altri dipendenti appartenenti allo stesso insieme.
\paragraph{Copertura oraria e turni}
Durante una giornata, l'obiettivo del responsabile di reparto è quello di coprire sempre o quasi il flusso della clientela in maniera adeguata. Rifacendoci al problema del \textit{Set Covering}, è possibile modellare la giornata lavorativa in diversi slot, i quali dovranno essere coperti da $f$ dipendenti, con $f$ o specificato dal caporeparto o individuato in seguito ad un'analisi delle vendite durante il corso dell'anno del supermercato. Definiamo quindi:
\begin{center}
	$f_{d,k}$ il fabbisogno per lo slot $k$ del giorno $d$.\\
	$K_d$ l'insieme degli slot del giorno $d$.
\end{center}
La finestra temporale che andiamo a modellizzare è di una settimana. Definiamo quindi:
\begin{center}
	$D=\{lun, \dots , dom\}=\{1,\dots ,7\}$ l'insieme dei giorni.
\end{center}
Gli slot temporali possono avere durata variabile, dall'intera giornata lavorativa fino a soli 15 minuti: abbiamo deciso di utilizzare uno slot temporale da mezz'ora. La giornata di lavoro inizia alle 08:00 e finisce alle 20:30, è quindi suddivisa in 25 slot temporali. L'orario è continuato ed è identico per tutti e sette i giorni della settimana, dunque vi sarà da definire un totale di 175 fabbisogni, ognuno per il singolo slot $k$ della giornata $d$.
I turni assegnati ai dipendenti saranno gli insiemi del \textit{Set Covering}. Specifichiamo:
\begin{center}
	$S=\{s_1, \dots , s_l\}$ l'insieme dei turni $s$
\end{center}
I turni $s$ sono caratterizzati da:
\begin{itemize}\label{defturno}
	\item $day(s)$: il giorno $d$ del turno $s$.
	\item $start(s)$: lo slot di inizio del turno $s$.
	\item $L_s$: durata ($\in \{6,7,\dots, 12\}$) del turno $s$.
	\item $A_{s,k}$: copertura dello slot $k$ dal turno $s$ (1 se coperto, 0 altrimenti).
\end{itemize}
Gli slot temporali dovranno essere coperti da almeno $f_{d,k}$ turni. Per definire se un lavoratore lavora o meno un certo turno, usiamo una variabile decisionale:
\begin{center}
	$x_{w,s}\in \{0,1\}$
\end{center}
la quale varrà 1 nel caso in cui al lavoratore $w$ sia assegnato il turno $s$, 0 altrimenti. Unendola alla matrice di copertura $A_{s,k}$ possiamo così dire se un lavoratore $w$, prendendosi in carico il turno $s$, copra o meno lo slot $k$.

I lavoratori si differenziano per il monte orario minimo in base alla tipologia di contratto:
\begin{center}
	$C_{w}\in \{20,24,48\}$ %rivedere
\end{center}
Una volta introdotto il necessario per l'orario e la copertura, possiamo iniziare a definire i vincoli relativi: 
\begin{align}
	& \sum_{w \in W}\sum_{s \in S : day(s)=d} A_{s,k} x_{w,s} 
	\ge f_{d,k} 
	&& \forall d, k \label{vincolo:copertura}\\
	& C_{w} \le \sum_{w \in W, s} 0.5*L_sx_{w,s} \le 48
	&& \text{monte ore} \label{vincolo:monteorept}
\end{align}
La copertura del fabbisogno si specifica mediante il primo vincolo. Questo vincolo ci assicura che almeno $f_{d,k}$ lavoratori siano presenti nello slot temporale $k$ del giorno $d$. Con questo vincolo non è ammissibile una situazione di difetto di personale, ma è ammissibile un eccesso di personale. Si potrebbe ammettere una situazione di difetto modificando il valore di $f_{d,k}$ o cambiando il peso del vincolo (introducendo ad esempio una nuova variabile $p_f$, il cui valore può ridurre o aumentare $f_{d,k}$). 

Ogni lavoratore deve lavorare almeno il monte orario minimo definito da contratto, ma può raggiungere le 48 ore facendo straordinari.\\
Gli ultimi vincoli riguardano il numero di turni giornalieri, che ricordiamo sono:
\begin{displayquote}
	\textit{Part time weekend: 20 ore settimanali, con disponibilità limitata ai giorni di venerdì, sabato e domenica.\\
	\dots \\
	I lavoratori full time possono coprire uno o due turni giornalieri, mentre i part time possono svolgerne solo uno.\\
	Il tempo massimo di lavoro giornaliero per un singolo dipendente è di 12 ore, anche se nella pratica si tende a non superare le 8 ore.\\}
\end{displayquote}
Si definiscono i seguenti vincoli:
\begin{align}
	& \sum_{w \in FT, \, s : day(s)=d} x_{w,s} \le 2, 
	&& \forall d \label{vincolo:ft}\\
	& \sum_{w \in PT, \, s : day(s)=d} x_{w,s} \le 1, 
	&& \forall d \label{vincolo:pt}\\
	& x_{w\in PTW,s} = 1, 
	&& day(s)=5 \land start(s)=16{:}30 \label{vincolo:weekend}\\[6pt]
	& x_{w\in PTW,s} = 0, 
	&& day(s)<5 \label{vincolo:weekend2}\\
	& \sum_{w \in PTW, \, s : day(s)=d} x_{w,s} \le 2, 
	&& d>5\label{vincolo:weekend3}
\end{align}
I primi due rappresentano i vincoli per i lavoratori full-time e part-time, mentre gli ultimi tre sono i vincoli per i part-time fine settimana. Per richiesta del datore, i lavoratori part-time fine settimana lavorano il turno della chiusura del venerdì, mentre il sabato e la domenica vengono trattati come i lavoratori full-time.\\
Si può infine introdurre la funzione obiettivo per questa parte:
\begin{equation}
	f_1=\sum_{w \in W}\sum_{s \in S} L_s x_{w,s} 
\end{equation}
ovvero la somma totale delle ore lavorate dall'intero roster dei cassieri, che andrà minimizzata come da richiesta.

\paragraph{Stacco turni}
I dipendenti hanno richiesto uno stacco tra due turni giornalieri il più breve possibile. Questo permetterebbe di ridurre al minimo il tempo trascorso sul luogo di lavoro e aumentare la soddisfazione del personale. Il problema si pone solo nel caso in cui il lavoratore faccia due turni in un giorno:
\begin{align}
	& split_{w,d} \ge \sum_{s : day(s)=d} x_{w,s} - 1\\
	& split_{w,d} \in \{0,1\}
\end{align}
è una variabile decisionale, che ci dice se il lavoratore $w$ è assegnato a due turni $s_1$ e $s_2$ nello stesso giorno $d$.\\
I turni $s$, per come sono \hyperref[defturno]{definiti}, permettono il calcolo dello slot di fine turno:
\begin{equation}
	end(s)=start(s)+L(s)
\end{equation}
e da questo posso trovare il valore dello stacco:
\begin{equation}
	stacco_{w,d} \ge  split_{w,d} \cdot (start(s_2) - end(s_1)), \quad \forall w \in W
\end{equation}
che viene considerato solo nel caso in cui effettivamente il lavoratore abbia due turni lo stesso giorno. Devono valere di conseguenza anche:
\begin{align*}
	& x_{w,s_1} = x_{w,s_2} = 1,\\
	& day(s_1) = day(s_2),\\
	& start(s_2) > end(s_1). 
\end{align*}
Bisogna infine minimizzare:
\begin{equation}
	f_2=\sum_{d \in D}\sum_{w \in W} stacco_{w,d}
\end{equation}
ovvero gli stacchi di tutti i lavoratori.

\paragraph{Giorni di riposo}
Si considera giorno di riposo un qualsiasi giorno della settimana nel quale il lavoratore non è stato assegnato a nessun turno:
\begin{equation}
	r_{w,d} \ge 1-\sum_{s\in S_d}x_{w,s}, \quad \forall w\in W \setminus PTW,\forall d\in D
\end{equation}
Durante la settimana, il lavoratore deve avere almeno un giorno di riposo e per evitare un sovraccarico di ore giornaliere può arrivare ad un massimo di due giorni di riposo settimanali:
\begin{equation}
	1 \le \sum_{d\in D} r_{w,d}\le 2
\end{equation}
Per i part-time fine settimana il problema non si pone, in quanto durante la settimana non gli si può assegnare turni.\\

\subsection{Problema esteso: prima parte}
\paragraph{Straordinari}
Gli straordinari vengono di molto penalizzati in quanto prevedono una retribuzione aggiuntiva, aumentando notevolemente il costo finale per il supermercato. Introduciamo quindi:
\begin{center}
	$O_w\in \mathbb{N}$
\end{center}
che rappresenta le ore straordinarie assegnate al lavoratore $w$. Questo valore è maggiore o uguale a 0 e si preferirebbe che valesse 0 per tutti i lavoratori del roster. Non vale 0 nel caso in cui:
\begin{align}
	& O_w \ge \sum_{s \in S : day(s)=d} A_{s,k} x_{w,s} - C_{w} 
	&& \forall w \in W \label{vincolo:straordinari}
\end{align}
ovvero al lavoratore $w$ vengono assegnate più ore di quelle previste dal contratto. La somma di questi valori va minimizzata:
\begin{equation}
	f_3=\sum_{w \in W}O_w
\end{equation}
Si potrebbe cambiare il peso di questo obiettivo nella funzione finale per così riadattarsi. Risulta particolarmente nei periodi festivi, quando la situazione diventa imprevedibile per malattie impreviste e flusso della clientela.

\paragraph{Equità}
I problemi riguardanti l'equità sono:
\begin{itemize}
	\item Carico orario settimanale uguale per tutti i lavoratori dello stesso tipo.
	\item Carico orario giornaliero uguale per tutti i lavoratori dello stesso tipo, ad esempio scatta una situazione di inequità nel momento in cui un lavoratore part-time lavori 6 ore un giorno mentre un altro ne lavora part-time solo 3.
	\item Nel caso dei lavoratori full-time e dei part-time, la gestione delle domeniche libere rientra nel principio di equità. 
\end{itemize}
Per il primo punto, introduciamo la variabile:
\begin{equation}
	H_w=\sum_{w \in T, \, s} 0.5*L_sx_{w,s}
\end{equation}
che rappresenta le ore totali assegnate al lavoratore $w$.\\
Introduciamo altre due variabili:
\begin{align}
	& d_{w}^{+}\ge 0,\quad d_{w}^{-}\le 0
	&& \forall w\in W\\
	& d_{w}^{+},d_{w}^{-}\in \mathbb{R}
\end{align}
che rappresentano rispettivamente le ore straordinarie oppure quelle mancanti per raggiungere il monte orario previsto da contratto. Vale quindi:
\begin{align}
	&d_{w}^{+}\ge H_w-C_w\\
	&d_{w}^{-}\le H_w-C_w
\end{align}
ove, se le ore assegnate sono di più rispetto a quelle previste, il valore di $d_{w}^{+}$ sarà uguale al surplus orario, mentre varrà il contrario per $d_{w}^{-}$ nel caso in cui venissero assegnate meno ore al dipendente. L'obiettivo è quello di avere uno stesso quantitativo orario per tutti i lavoratori dello stesso tipo, quindi si minimizza:
\begin{equation}
	f_4=\sum_{T\in W}\sum_{w\in T} \alpha_1 d_{w}^{+} + \beta_1 d_{w}^{-}
\end{equation}
cioè la somma delle deviazioni dal valore previsto da contratto per ogni tipo. Vale $\alpha \ge 0$, $\beta \le 0$ e in base a quale situazione sia preferibile (di eccesso o di difetto) posso decidere tra $\alpha \ge |\beta|$ oppure $\alpha < |\beta|$. Nel nostro caso consideriamo $\alpha > |\beta|$, in quanto una situazione di surplus è sfavorita per scelta del supermercato.\\
Altro problema è la durata dei turni. Sono permessi turni da 3 fino a 6 ore, ma una durata eccessiva dei turni non è ben vista dal personale e potrebbe portare a situazioni di squilibrio in cui alcuni dipendenti lavorano un solo turno da 6 ore al giorno mentre altri ne lavorano due da 3/4 ore ciascuno. Per questione di praticità, possiamo imporre un'unica preferenza per l'intero personale verso il turno da 4 ore. Introduciamo quindi:
\begin{align}
	&dur_{s}^{+}\ge 0,\quad dur_{s}^{-}\le 0
	&& \forall s\in S \\
	& dur_{s}^{+}, dur_{s}^{-}\in \mathbb{R}
\end{align}
che indicano la durata di tutti i turni $s$ attivi, aggiungiamo anche:
\begin{align}
	&dur_{s}^{+}\ge 0.5*L_s x_{w,s} - 4.0 \\
	&dur_{s}^{-}\le 0.5*L_s x_{w,s} - 4.0
\end{align}
e, infine, da minimizzare:
\begin{equation}
	f_5=\sum_{s \in S} \alpha_2 dur_{s}^{+} + \beta_2 dur_{s}^{-}
\end{equation}
Ogni dipendente potrebbe esprimere la propria preferenza e si potrebbe risolvere il problema andando a cambiare i parametri appena visti. Come prima, $\alpha \ge 0$, $\beta \le 0$ e $\alpha > |\beta|$, in quanto un turno che superi le 4 ore risulta molto più pesante rispetto ad un turno di 3 ore.
\paragraph{Riposo continuato}
Si introduce una nuova variabile binaria per indicare il riposo continuato per i lavoratori full-time:
\begin{gather}
	c_{w,d}\le r_{w,d}\\
	c_{w,d}\le r_{w,d+1}\\
	c_{w,d}\le r_{w,d} + r_{w,d+1} - 1\\
	c_{w,d}\in \{0,1\}, \forall w\in W \setminus PTW, \forall d\in D
\end{gather}
che vale 1 solo nel caso in cui il lavoratore $w$ ha due giorni di riposo consecutivi $d$ e $d+1$ durante la settimana. Si vuole massimizzare questa quantità per preferenza del personale:
\begin{equation}
	f_6=\sum_{w\in FT}\sum_{d\in D}c_{w,d}
\end{equation}

\subsection{Funzione obiettivo}
\begin{align}
    \min \; & 
    \sum_{i=1}^6\lambda_if_i
    && \lambda_i \in \mathbb{R}
\end{align}
I $\lambda_i$ rappresentano i pesi che noi diamo ai vari obiettivi: più $lambda_i$ è elevato, più l'obiettivo relativo incide nella funzione obiettivo generale, ottenendo di fatto più importanza nella ricerca dell'ottimo. 

